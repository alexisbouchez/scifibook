% ============================================================================
%                      CHAPITRE 9 — LA SURFACE
% ============================================================================

\chapter{La Surface}

\bigskip

Kowalski franchit le sas et s'enfonça dans le vide.

Depuis le poste d'observation, Élise suivait sa progression sur les écrans de contrôle. La silhouette blanche de l'ingénieur, sanglée dans sa combinaison EVA, dérivait lentement vers la masse noire de l'Objet. Les propulseurs de son scaphandre crachaient de brèves bouffées de gaz, ajustant sa trajectoire avec une précision millimétrée.

«\,Distance : cinq cents mètres\,», annonça ARIA. «\,Tous les paramètres vitaux sont nominaux.\,»

«\,Je confirme\,», dit la voix de Kowalski dans le communicateur. «\,Approche nominale. Je distingue la surface maintenant. C'est...\,» Il s'interrompit un instant. «\,C'est très noir.\,»

«\,Plus précis, Kowalski\,», demanda Vasquez depuis le poste de pilotage.

«\,Je veux dire vraiment noir. Il n'y a pas de reflet, pas de texture. On dirait un trou dans l'espace. Comme si quelqu'un avait découpé un morceau du vide et l'avait remplacé par... rien.\,»

Élise comprenait ce qu'il décrivait. Les images ne rendaient pas justice à cette noirceur absolue, à cette absence totale de lumière renvoyée. L'Objet n'était pas simplement sombre ; il était l'antithèse même de la visibilité.

«\,Distance : deux cents mètres. Je ralentis l'approche.\,»

Sur l'écran, la silhouette de Kowalski s'immobilisait presque, dérivant maintenant à une vitesse infinitésimale vers la surface courbe. Élise retenait son souffle. Dans quelques minutes, un être humain toucherait pour la première fois un artefact extraterrestre. Trois décennies de carrière, une vie entière de questions, convergeaient vers cet instant.

«\,Cinquante mètres. Je commence à sentir une attraction gravitationnelle locale.\,»

«\,Confirmé\,», dit ARIA. «\,Le champ gravitationnel de l'Objet est légèrement supérieur aux prédictions. Ajustement de la trajectoire recommandé.\,»

«\,Je compense. Vingt mètres.\,»

Silence. Dans le poste d'observation, personne ne bougeait. Okonkwo avait joint les mains devant elle, dans un geste qui ressemblait à une prière ; Vasquez fixait les écrans avec une intensité féroce ; Élise sentait son cœur battre dans ses tempes.

«\,Dix mètres. Je tends le bras.\,»

La caméra embarquée sur le casque de Kowalski montrait sa main gantée qui s'avançait vers la surface noire. Derrière, les étoiles brillaient avec une indifférence éternelle.

«\,Contact.\,»

Le mot résonna dans le silence du vaisseau. Élise vit, sur l'écran, la main de Kowalski se poser sur l'Objet — ou plutôt, disparaître dans sa noirceur, comme avalée par l'absence de lumière.

«\,La surface est... solide\,», dit Kowalski. «\,Lisse. Pas de texture perceptible. Température... ARIA, tu reçois les données\,?\,»

«\,Je reçois. Température de surface : trois kelvin au-dessus du zéro absolu. Aucune émission thermique détectable.\,»

«\,C'est impossible\,», murmura Okonkwo. «\,À cette distance du Soleil, la température devrait être d'au moins quarante kelvin.\,»

«\,L'Objet absorbe presque toute la lumière qu'il reçoit\,», dit Élise. «\,Et apparemment, il ne rayonne pas non plus. C'est comme s'il...\,» Elle chercha le mot juste. «\,Comme s'il gardait tout pour lui.\,»

Kowalski continua son exploration, se déplaçant le long de la surface courbe avec des mouvements prudents. Sa main gantée caressait la paroi noire, cherchant une aspérité, une jointure, n'importe quoi qui pût trahir une structure sous-jacente.

«\,C'est parfaitement uniforme\,», rapporta-t-il. «\,Pas la moindre variation. On dirait que la surface entière a été coulée d'un seul bloc.\,»

«\,Continuez sur cinquante mètres\,», ordonna Vasquez. «\,Si vous ne trouvez rien, vous rentrez.\,»

Kowalski obéit, progressant lentement le long de l'équateur invisible de la sphère. Les minutes s'étiraient, interminables. Élise scrutait l'écran, cherchant elle aussi le moindre indice, la moindre anomalie dans cette perfection monotone.

Puis Kowalski s'arrêta.

«\,Attendez\,», dit-il. «\,Il y a quelque chose ici.\,»

L'écran montra sa main qui s'immobilisait sur un point précis de la surface. À première vue, il n'y avait rien — juste la même noirceur absolue qu'ailleurs. Mais en plissant les yeux, Élise crut distinguer quelque chose. Une variation infime dans l'absence de lumière, un motif presque imperceptible.

«\,ARIA, amélioration d'image\,», demanda-t-elle.

L'IA obéit, et le motif se précisa. Des lignes, tracées dans la surface noire — si fines qu'elles étaient presque invisibles, mais indéniablement présentes. Elles formaient une figure géométrique complexe, une sorte de rosace aux symétries multiples.

«\,C'est artificiel\,», souffla Okonkwo. «\,C'est définitivement artificiel.\,»

Élise hocha la tête, le cœur battant. Ce n'était pas une erreur. Ce n'était pas une illusion. Quelqu'un avait gravé ce motif dans la surface de l'Objet, il y avait des milliards d'années, pour que quelqu'un d'autre le trouve un jour.

«\,Kowalski\,», dit-elle, «\,pouvez-vous cartographier le motif complet\,?\,»

«\,Je vais essayer. Ça va prendre du temps — c'est très grand.\,»

«\,Prenez tout le temps qu'il faut. Nous venons peut-être de trouver la porte.\,»
