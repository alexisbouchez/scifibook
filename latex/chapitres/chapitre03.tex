% ============================================================================
%                      CHAPITRE 3 — LE SIGNAL
% ============================================================================

\chapter{Le Signal}

\bigskip

Les images arrivèrent un mardi, à quatorze heures trente-sept, heure de Paris.

Élise se trouvait au Centre européen d'opérations spatiales de Darmstadt, dans cette vaste salle aux allures de cathédrale technologique où des dizaines d'écrans projetaient en temps réel les données des missions en cours. Elle avait pris le TGV la veille au soir, incapable de supporter l'idée de recevoir les images par transmission différée, et avait passé la nuit dans un hôtel médiocre près de la gare, à contempler le plafond en attendant l'aube.

Kepler-III, le plus puissant télescope spatial jamais construit par l'humanité, avait braqué son miroir de trente mètres vers les coordonnées qu'elle avait fournies. Pendant quarante-huit heures, ses capteurs infrarouges avaient accumulé les photons venus des confins du système solaire, construisant pixel par pixel l'image de ce qui s'y dissimulait.

Lorsque le fichier s'afficha sur l'écran principal, Élise sentit le sol se dérober sous ses pieds.

C'était là.

Une sphère parfaite, d'un noir si absolu qu'elle semblait moins refléter la lumière que l'absorber, se découpait sur le fond étoilé de la Voie lactée. Les algorithmes de traitement avaient calculé son diamètre avec une précision de quelques mètres : douze kilomètres et trois cent quarante-sept mètres. Pas une irrégularité, pas une aspérité, pas le moindre relief. Une surface lisse comme un miroir, courbe comme une larme figée dans l'éternité du vide.

Autour d'elle, les techniciens du centre continuaient leurs activités routinières, inconscients de ce qui venait d'apparaître sur l'écran d'Élise. Elle avait demandé une station de travail isolée, dans un coin de la salle, et personne ne prêtait attention à cette femme grisonnante qui fixait son moniteur avec l'intensité d'un mystique face à une apparition divine.

Elle agrandit l'image, zooma sur les bords de la sphère, chercha en vain une texture, un défaut, quelque chose qui pût trahir une origine naturelle. Il n'y avait rien. L'objet était aussi parfait qu'une abstraction mathématique — un idéal platonicien incarné dans la matière.

Puis elle remarqua autre chose.

Dans le coin inférieur de l'écran, un indicateur clignotait : les radioastronomes du réseau ALMA, en complément des observations visuelles, avaient effectué un balayage spectral de la région. Et ils avaient trouvé quelque chose.

Élise ouvrit le fichier audio, brancha ses écouteurs, et le monde autour d'elle cessa d'exister.

Un son. Un son venu de l'Objet.

Ce n'était pas le silence du vide qu'elle s'attendait à entendre, ni le bruit blanc des radiations cosmiques. C'était une séquence — régulière, répétitive, d'une pureté mathématique qui ne pouvait être le fruit du hasard. Elle écouta une fois, puis deux, puis dix, transcrivant mentalement les impulsions en nombres.

\textit{Un. Silence. Deux. Silence. Trois. Silence. Cinq. Silence. Sept. Silence. Onze. Silence. Treize...}

Les nombres premiers.

La séquence des nombres premiers, transmise sur 1420 mégahertz — la fréquence de l'hydrogène, cette signature universelle que tout astronome connaissait sous le nom de «\,canal de l'eau\,». C'était la fréquence que l'humanité elle-même avait choisie pour ses programmes de recherche d'intelligence extraterrestre, celle que Sagan et Drake avaient identifiée, un siècle plus tôt, comme le point de rendez-vous naturel de toute civilisation technologique.

Et quelqu'un — quelque chose — émettait sur cette fréquence, depuis les ténèbres de la Ceinture de Kuiper.

Élise ôta ses écouteurs d'une main tremblante. Son cœur battait si fort qu'elle l'entendait pulser dans ses tempes. Elle avait lu tous les ouvrages sur le sujet, toutes les spéculations des théoriciens, tous les scénarios imaginés par les optimistes et les pessimistes. Elle savait ce que signifiait cette séquence — ce qu'elle avait toujours été censée signifier si jamais on la détectait un jour.

Ce n'était pas un signal naturel. Ce n'était pas une coïncidence. Ce n'était pas une erreur.

C'était un message.

Un message qui disait, dans le seul langage véritablement universel : \textit{Nous sommes ici. Nous savons compter. Nous pensons.}

Elle resta immobile pendant ce qui lui parut une éternité, les yeux rivés sur l'image de la sphère noire qui occupait l'écran. Quelque part dans son esprit, une voix rationnelle lui soufflait qu'elle devait garder son calme, qu'elle devait vérifier, revérifier, s'assurer qu'il ne s'agissait pas d'une contamination terrestre ou d'une anomalie instrumentale. Mais une autre voix, plus ancienne et plus profonde, lui murmurait qu'elle savait. Qu'elle avait toujours su, depuis cette première nuit à l'Observatoire, que l'univers venait de lui répondre.

Elle décrocha le téléphone sécurisé du centre et composa le numéro direct de la directrice générale de l'ESA.

«\,Madame la Directrice\,», dit-elle d'une voix qu'elle s'efforça de garder neutre, «\,je crois que vous devriez voir quelque chose. Immédiatement.\,»

\bigskip

Quarante-huit heures plus tard, une réunion d'urgence était convoquée au siège parisien de l'Agence. Douze personnes seulement — les directeurs des principaux programmes, le chef de la communication, trois experts en exobiologie, et Élise elle-même — se rassemblèrent dans une salle de conférence dont les fenêtres avaient été obturées et les communications externes coupées.

Le secret, pour l'instant, tenait encore. Mais Élise savait qu'il ne tiendrait plus très longtemps.

L'univers avait parlé. L'humanité allait devoir répondre.
