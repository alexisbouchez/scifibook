% ============================================================================
%                     CHAPITRE 12 — L'INTÉRIEUR
% ============================================================================

\chapter{L'Intérieur}

\bigskip

L'architecture défiait l'entendement.

Élise et Kowalski avancèrent pendant ce qui leur parut des heures, bien que leurs chronographes indiquassent seulement quarante-sept minutes. Le corridor s'était mué en une succession de salles aux géométries impossibles — des espaces où les angles ne s'additionnaient pas correctement, où les perspectives se tordaient de manières qui faisaient mal aux yeux. ARIA, d'habitude si précise dans ses analyses, avouait son impuissance.

«\,Les mesures sont incohérentes\,», rapportait l'IA. «\,La distance parcourue selon mes capteurs ne correspond pas à la distance théorique selon la courbure de l'Objet. C'est comme si l'espace intérieur était... plié.\,»

«\,Plus grand à l'intérieur qu'à l'extérieur\,», murmura Élise.

C'était une idée qu'elle avait rencontrée dans les théories les plus spéculatives de la physique — des espaces repliés sur eux-mêmes, des dimensions supplémentaires accessibles par des manipulations gravitationnelles. Mais voir cette théorie incarnée, marcher à l'intérieur d'une structure qui la confirmait, était une chose entièrement différente.

Les salles qu'ils traversaient n'étaient pas vides. Des structures s'y dressaient — piliers, consoles, formes géométriques dont la fonction demeurait obscure. Tout était fait de la même matière noire, animée des mêmes veines lumineuses, mais chaque élément semblait avoir un but précis qu'Élise ne parvenait pas à identifier.

«\,Regardez ça\,», dit Kowalski en s'approchant d'une formation particulière.

C'était une sorte de piédestal, surmonté d'une sphère creuse faite de cercles concentriques imbriqués. Quand Kowalski approcha sa main gantée, les cercles se mirent à tourner, chacun à une vitesse différente, créant un ballet hypnotique de mouvements entrelacés.

«\,Un mécanisme\,?\,» hasarda-t-il.

«\,Ou une représentation\,», dit Élise. «\,Les cercles... ils pourraient représenter des orbites. Un système solaire miniature.\,»

Elle compta les anneaux. Huit. Comme les huit planètes. Mais la sphère centrale était vide — pas de soleil au milieu, juste un espace creux où quelque chose aurait dû se trouver.

«\,Il manque quelque chose\,», dit-elle.

Ils continuèrent. Les salles se succédaient, chacune plus étrange que la précédente. Dans l'une, des cristaux translucides flottaient en lévitation, émettant une lumière qui changeait de couleur quand on s'approchait. Dans une autre, les murs étaient couverts de ce qui ressemblait à des inscriptions — des symboles qui n'appartenaient à aucun langage humain, gravés dans la matière noire avec une précision microscopique.

«\,Thulé, vous recevez toujours\,?\,» demanda Élise.

«\,On reçoit\,», confirma la voix d'Okonkwo. «\,Mais le signal est dégradé. On perd des détails.\,»

«\,Compris. Nous continuons.\,»

Ils atteignirent enfin une salle plus vaste que les autres — une cavité sphérique, peut-être, bien que les dimensions fussent difficiles à estimer dans cet espace distordu. Au centre flottait une structure qui fit s'arrêter Élise net.

C'était un polyèdre irrégulier, composé de centaines de faces dont chacune affichait une image différente. Des paysages, comprit-elle en s'approchant. Des mondes. Des ciels étrangers, des horizons impossibles, des géographies qui n'existaient nulle part dans le système solaire.

«\,Ce sont des enregistrements\,», murmura-t-elle. «\,Des images de... d'ailleurs.\,»

Une face montrait un désert rouge sous un ciel violet, parsemé de formations rocheuses qui défiaient la gravité. Une autre présentait un océan d'un bleu profond, surmonté de trois lunes qui s'alignaient à l'horizon. Une troisième révélait une forêt — si l'on pouvait appeler ainsi ces structures végétales tentaculaires — baignée d'une lumière dorée.

«\,Ce sont les mondes qu'ils ont visités\,», dit Kowalski. «\,Ceux qui ont construit l'Objet. Ils ont voyagé, et ils ont enregistré ce qu'ils ont vu.\,»

Élise hocha la tête, fascinée. Des milliers de faces, des milliers de mondes. Une civilisation qui avait parcouru la galaxie, peut-être au-delà, et qui avait tout consigné ici, dans cette structure impossible.

Mais quelque chose la troublait.

«\,Kowalski\,», dit-elle lentement, «\,regardez les images. Vraiment. Qu'est-ce que vous voyez\,?\,»

L'ingénieur examina les faces du polyèdre, passant de l'une à l'autre avec attention. Puis son visage changea.

«\,Il n'y a personne\,», dit-il. «\,Des paysages, des bâtiments, des structures... mais pas de vie. Pas d'êtres vivants.\,»

C'était vrai. Chaque image montrait un monde apparemment désert — des civilisations abandonnées, des cités vides, des planètes mortes. Même les végétations semblaient figées, fossilisées dans une éternité sans mouvement.

«\,Ils ont cherché\,», comprit Élise. «\,Pendant des millions d'années, peut-être. Ils ont parcouru l'univers en cherchant d'autres formes de vie, d'autres intelligences. Et ils n'ont trouvé que... ça.\,»

Des échos. Des ruines. Des souvenirs de civilisations disparues.

La solitude cosmique, incarnée dans des milliers d'images.
