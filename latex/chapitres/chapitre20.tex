% ============================================================================
%                 CHAPITRE 20 — LE VOYAGE DU RETOUR
% ============================================================================

\chapter{Le Voyage du Retour}

\bigskip

Les huit mois du retour furent à la fois interminables et trop courts.

Le \textit{Thulé} s'éloigna de l'Objet par une nuit de décembre — si l'on pouvait appeler ainsi un moment arbitraire dans l'éternité sans saisons de l'espace profond. Élise regarda la sphère noire rétrécir dans le hublot arrière, jusqu'à ce qu'elle ne soit plus qu'un point parmi les étoiles, indistinguable du reste de l'univers.

Elle avait changé. Elle le savait, et les autres le voyaient aussi.

Ce n'était pas quelque chose de visible — pas une transformation physique, pas une marque sur son corps. C'était dans ses yeux, disait Okonkwo. Dans la façon dont elle regardait les étoiles maintenant. Dans les silences qu'elle laissait s'étirer avant de répondre aux questions.

Les premières semaines, elle dormit beaucoup. Son corps récupérait de l'épreuve de la connexion, et son esprit avait besoin de temps pour intégrer ce qu'elle avait reçu. Des bribes de mémoires aliens surgissaient parfois dans ses rêves — des paysages qu'elle n'avait jamais vus, des émotions qu'elle n'avait jamais ressenties, des pensées formulées dans des langages qui n'avaient pas de mots.

Puis, progressivement, elle commença à parler.

Avec Okonkwo d'abord, la plus réceptive de l'équipage. Elles passaient des heures dans le module d'observation, contemplant le vide constellé, discutant de ce que les constructeurs avaient révélé.

«\,Comment fait-on\,?\,» demanda Okonkwo un soir. «\,Comment dit-on à l'humanité qu'elle est probablement seule dans l'univers\,?\,»

«\,On ne le dit pas comme ça\,», répondit Élise. «\,Ce n'est pas le message. Le message, c'est que notre solitude nous rend précieux. Que chaque conscience est un miracle. Que regarder les étoiles et se demander pourquoi, c'est la chose la plus importante que nous puissions faire.\,»

«\,Vous croyez que les gens comprendront\,?\,»

Élise hésita.

«\,Certains, oui. D'autres non. Certains seront terrifiés, d'autres déprimés. Il y aura des crises de foi, des remises en question, de la colère. Mais il y aura aussi...\,» Elle chercha le mot juste. «\,De l'émerveillement. De la gratitude. Une nouvelle façon de regarder notre existence.\,»

Avec Kowalski, les conversations étaient différentes — plus pratiques, plus terre-à-terre. L'ingénieur voulait savoir comment fonctionnait l'Objet, quels principes physiques permettaient de replier l'espace, de maintenir une structure pendant des milliards d'années.

«\,Je n'ai pas de réponses techniques\,», avoua Élise. «\,Ce qu'ils m'ont transmis n'était pas de la science au sens où nous l'entendons. C'était... une vision du monde. Une philosophie.\,»

«\,Une philosophie ne fait pas voler les vaisseaux\,», grommela Kowalski.

«\,Non. Mais elle peut donner une raison de les construire.\,»

Vasquez, elle, gardait ses distances. La commandante accomplissait son devoir — maintenir le vaisseau en état, superviser le voyage, préparer le rapport de mission — mais quelque chose s'était brisé entre elle et Élise depuis l'incident du cristal. La confiance, peut-être. Ou simplement la compréhension mutuelle.

Ce fut seulement dans le dernier mois, alors que la Terre commençait à grandir dans les hublots, que Vasquez vint la trouver.

«\,Je vous ai jugée trop durement\,», dit-elle sans préambule. «\,Vous avez pris un risque insensé, mais... je comprends pourquoi.\,»

Élise la regarda, surprise.

«\,Vraiment\,?\,»

«\,Vous êtes venue ici pour trouver des réponses. Toute votre vie, vous avez cherché des réponses. Comment auriez-vous pu refuser la dernière porte\,?\,» Vasquez eut un sourire las. «\,Je suis pilote, Morneau. Je vis pour le contrôle, la prudence, la maîtrise des risques. Vous êtes scientifique. Vous vivez pour l'inconnu.\,»

«\,Et maintenant\,? Maintenant que nous savons ce que nous savons\,?\,»

Vasquez regarda par le hublot, vers cette petite bille bleue qui grossissait de jour en jour.

«\,Maintenant, on rentre à la maison. Et on laisse l'humanité décider ce qu'elle fait de tout ça.\,»
