% ============================================================================
%                    CHAPITRE 1 — LA DÉCOUVERTE
% ============================================================================

\chapter{La Découverte}

\epigraph{«\,Il est des nuits où l'univers consent enfin à murmurer ses secrets à ceux qui veillent.\,»}{}

\bigskip

La nuit avait depuis longtemps englouti Paris lorsque Élise Morneau releva les yeux de son écran pour la première fois en quatre heures. À travers la haute fenêtre de son bureau, nichée sous les combles de l'Observatoire, elle apercevait les lumières de la ville qui scintillaient comme une pâle imitation des étoiles qu'elle ne pouvait plus voir. La pollution lumineuse, ce fléau des astronomes, avait depuis un siècle chassé le ciel nocturne des capitales ; mais les données, elles, n'avaient que faire de l'obscurité.

Elle s'étira, sentit craquer les vertèbres de sa nuque endolorie, et consulta machinalement la montre mécanique qu'elle portait au poignet gauche — un héritage de son père, ingénieur chez Arianespace, mort vingt ans plus tôt dans l'effondrement d'un hangar d'assemblage. Deux heures quarante-sept. Janvier avait figé les rues désertes dans un silence que même le ronronnement des serveurs ne parvenait pas à troubler.

Elle aurait dû rentrer. Léa dormait seule dans l'appartement du quinzième arrondissement, et le message qu'Élise lui avait envoyé à vingt-deux heures — «\,Je serai là avant minuit\,» — s'était mué en promesse rompue. Une de plus. À seize ans, sa fille avait cessé de compter ces déceptions ; elle s'était murée dans une indifférence polie qui blessait Élise plus profondément que n'importe quelle colère.

Mais les données d'Horizon-7 ne pouvaient pas attendre.

La sonde, lancée onze ans auparavant vers les confins du système solaire, avait dépassé l'orbite de Neptune l'année précédente et s'enfonçait maintenant dans les ténèbres de la Ceinture de Kuiper. Élise ne s'intéressait pas à sa mission principale — la cartographie des objets transneptuniens pour le compte de l'Agence Spatiale Européenne — mais aux perturbations gravitationnelles que ses instruments enregistraient avec une précision d'un milliardième de mètre par seconde carrée. C'était dans ces infimes variations, ces frémissements du tissu spatial, qu'elle espérait trouver la confirmation de ses théories sur la distribution de masse dans les régions externes du système.

Ce qu'elle avait trouvé, en revanche, n'avait rien à voir avec ses théories.

Les chiffres s'alignaient sur l'écran, colonnes après colonnes de données brutes que le logiciel d'analyse convertissait en courbes et en graphiques. Élise avait d'abord cru à une erreur instrumentale — un dysfonctionnement des accéléromètres, peut-être, ou une corruption des données lors de leur transmission à travers les quatre milliards de kilomètres qui séparaient Horizon-7 de la Terre. Elle avait relancé les vérifications, appliqué les protocoles de correction, filtré le bruit statistique avec une rigueur obsessionnelle.

L'anomalie persistait.

Quelque chose, là-bas, dans le noir absolu qui s'étendait au-delà de Pluton, exerçait une attraction gravitationnelle que rien ne pouvait expliquer.

Élise se pencha de nouveau sur l'écran, le front plissé par cette concentration intense qui, lui avait-on dit un jour, la faisait ressembler à une chouette mécontente. Elle fit défiler les données télémétriques, croisa les mesures avec les éphémérides des corps connus, élimina méthodiquement chaque explication rationnelle. Ce n'était pas Éris. Ce n'était pas Makémaké. Ce n'était aucun des milliers d'objets répertoriés dans les catalogues du système solaire externe.

C'était autre chose.

Les calculs orbitaux, qu'elle reprit trois fois de suite en variant les paramètres initiaux, convergeaient vers une conclusion impossible : un objet d'environ douze kilomètres de diamètre, situé à quarante-sept unités astronomiques du Soleil, en orbite parfaitement circulaire — une circularité si absolue que la probabilité qu'elle résulte d'un processus naturel était, selon ses estimations, inférieure à un sur dix milliards.

La nature ne produisait pas de cercles parfaits.

Élise sentit un frisson lui parcourir l'échine, et ce n'était pas le froid de janvier qui s'infiltrait par les interstices des vieilles fenêtres. Elle connaissait cette sensation — ce mélange d'excitation et de terreur qui accompagnait les moments où le voile de l'univers semblait sur le point de se déchirer. Elle l'avait éprouvée deux fois auparavant : le jour où elle avait compris, à vingt-trois ans, que sa thèse sur la migration planétaire était correcte ; et le jour où le médecin lui avait annoncé que Marc, son mari, ne verrait pas le printemps.

Dans les deux cas, le monde avait basculé.

Elle s'empara d'un carnet — elle n'avait jamais renoncé au papier pour les moments importants — et nota de sa petite écriture serrée les coordonnées de l'objet, sa masse estimée, les paramètres de son orbite. Puis elle resta immobile, le stylo suspendu au-dessus de la page, tandis que les implications de sa découverte se déployaient dans son esprit comme les ondes concentriques d'un caillou jeté dans l'eau.

Si ces données étaient exactes — et tout indiquait qu'elles l'étaient — alors quelque chose d'artificiel flottait aux confins du système solaire. Quelque chose qu'aucune main humaine n'avait façonné. Quelque chose qui attendait, depuis des milliards d'années peut-être, que quelqu'un le remarque.

«\,Ce n'est pas possible\,», murmura-t-elle dans le silence du bureau.

Mais les chiffres ne mentaient pas. Les chiffres ne connaissaient ni l'espoir ni la peur. Ils se contentaient d'être, avec cette indifférence absolue des mathématiques face aux désirs humains.

Élise ferma les yeux un instant, inspira profondément, puis les rouvrit sur l'écran qui luisait doucement dans la pénombre. Elle avait cinquante-deux ans. Elle avait consacré trois décennies de sa vie à scruter l'univers, à traquer ses anomalies, à déchiffrer ses murmures. Elle avait sacrifié son mariage — non, corrigea-t-elle intérieurement, le cancer avait pris Marc, pas la science — puis sa relation avec sa fille, puis toute prétention à une vie normale, pour ces moments exactement.

Ces moments où l'univers daignait enfin répondre.

Elle sauvegarida les données sur trois supports différents, envoya une copie cryptée vers les serveurs de l'ESA à Darmstadt, et resta encore une heure à contempler les courbes qui dansaient sur l'écran. Dehors, Paris dormait, ignorante de ce qui venait peut-être de changer pour toujours. Et quelque part, à des milliards de kilomètres de là, dans un froid si absolu que même la lumière peinait à s'y aventurer, un objet impossible tournait en silence autour du Soleil.

Il attendait depuis plus longtemps que la Terre elle-même n'existait.

Il n'aurait plus très longtemps à attendre.
