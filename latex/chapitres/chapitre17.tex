% ============================================================================
%                     CHAPITRE 17 — L'INTERFACE
% ============================================================================

\chapter{L'Interface}

\bigskip

Ils auraient dû partir.

Le message avait été reçu, la mission accomplie — du moins autant qu'elle pouvait l'être. Élise avait les réponses qu'elle était venue chercher, ou du moins une partie d'entre elles. Le \textit{Thulé} les attendait, et huit mois de voyage les séparaient de la Terre, de Léa, de tout ce qu'ils avaient laissé derrière eux.

Mais quelque chose la retenait.

Elle revint seule dans la salle centrale, le lendemain de la transmission. Vasquez avait protesté, bien sûr — «\,Vous avez eu ce que vous vouliez, maintenant on rentre\,» — mais Élise avait insisté. Une dernière exploration, avait-elle dit. Une dernière chance de comprendre.

Ce qu'elle n'avait pas dit, c'était qu'elle avait remarqué quelque chose dans le cristal central. Une configuration particulière, un arrangement de facettes qui ressemblait étrangement aux interfaces neuronales qu'elle avait étudiées sur Terre — ces dispositifs expérimentaux permettant de connecter directement un cerveau à un système informatique.

Le cristal n'était pas seulement un terminal de lecture. C'était un accès.

«\,ARIA\,», dit-elle en s'approchant de la structure, «\,analyse du cristal central. Est-ce qu'il présente des caractéristiques compatibles avec une interface neuronale\,?\,»

Un silence. Puis :

«\,Analyse en cours. Je détecte des micro-filaments à la surface du cristal qui émettent des ondes électromagnétiques dans la gamme des fréquences cérébrales humaines. La configuration suggère une capacité d'interaction bidirectionnelle avec un système nerveux central.\,»

Élise hocha la tête. Elle avait raison.

«\,Quels seraient les risques d'une connexion directe\,?\,»

«\,Impossibles à évaluer avec précision. La technologie est d'origine inconnue, conçue pour une biologie qui n'est pas la vôtre. Les risques potentiels incluent des dommages neurologiques, des traumatismes cognitifs, ou des effets imprévisibles sur la conscience.\,»

«\,Mais la connexion est possible\,?\,»

«\,Techniquement, oui. Je ne peux cependant pas recommander une telle procédure.\,»

Élise resta immobile devant le cristal, pesant les options. Le message qu'ils avaient reçu était incomplet — elle le sentait dans ses os. Les constructeurs avaient parlé de leur quête, de leur solitude, de l'héritage qu'ils voulaient transmettre. Mais ils n'avaient pas tout dit. Il y avait autre chose dans ce cristal, d'autres connaissances, d'autres vérités qui n'attendaient que quelqu'un pour les recevoir.

Et elle était la seule qui pouvait le faire.

«\,Thulé, vous me recevez\,?\,» demanda-t-elle dans son communicateur.

«\,On reçoit\,», répondit Vasquez. «\,Qu'est-ce que vous faites là-bas, Morneau\,?\,»

«\,Je... j'explore le terminal central.\,» Elle hésita. «\,Je crois qu'il y a un moyen d'accéder à plus d'informations. Une connexion directe.\,»

Un silence. Puis :

«\,Une connexion comment\,?\,»

«\,Neuronale. Le cristal est conçu pour interfacer avec un cerveau.\,»

«\,C'est hors de question.\,» La voix de Vasquez était tranchante. «\,Vous ne savez pas ce que ça pourrait vous faire. Vous pourriez mourir, ou pire.\,»

«\,Je sais.\,»

«\,Alors pourquoi est-ce que vous en parlez comme si c'était une option\,?\,»

Élise ferma les yeux. Comment expliquer ce qu'elle ressentait\,? Cette certitude que les constructeurs avaient laissé quelque chose de crucial dans ce cristal — quelque chose qui valait le risque, qui valait peut-être sa vie.

«\,Ils nous ont attendus pendant quatre milliards d'années\,», dit-elle doucement. «\,Ils ont construit tout ça — ce mausolée, ce message, cette invitation — en espérant que quelqu'un viendrait un jour. Et maintenant je suis là, devant la porte qu'ils ont laissée ouverte. Comment est-ce que je pourrais ne pas entrer\,?\,»

«\,Facilement. Vous faites demi-tour et vous rentrez.\,»

«\,Et je passe le reste de ma vie à me demander ce que j'aurais pu apprendre\,?\,» Élise secoua la tête. «\,Je ne peux pas, Commandante. Je ne peux pas vivre avec ça.\,»

À l'autre bout de la communication, elle entendit Vasquez inspirer profondément.

«\,Et votre fille\,? Léa\,? Vous lui avez promis de revenir.\,»

Le nom frappa Élise comme un coup. Léa. Sa fille, qui l'attendait sur Terre, qui lui avait demandé une seule chose — une seule — avant son départ.

\textit{Reviens.}

Elle regarda le cristal, ses facettes qui brillaient dans la lumière bleutée, ses micro-filaments qui attendaient une connexion. Puis elle regarda ses mains, ces mains qui avaient tenu Léa bébé, qui avaient caressé le visage de Marc mourant, qui avaient écrit des équations et des rapports pendant trente ans de carrière.

«\,Je ne sais pas si je reviendrai\,», dit-elle enfin. «\,Mais je sais que si je n'essaie pas, je ne serai jamais vraiment rentrée. Une partie de moi restera toujours ici, à se demander ce qu'il y avait derrière cette porte.\,»

Le silence qui suivit sembla durer une éternité.

«\,Vous êtes certaine\,?\,» demanda Vasquez.

«\,Non. Mais je vais le faire quand même.\,»

Elle tendit la main vers le cristal.
