% ============================================================================
%                     CHAPITRE 15 — LE CENTRE
% ============================================================================

\chapter{Le Centre}

\bigskip

Le cœur de l'Objet était un tombeau.

Élise le comprit dès l'instant où ils franchirent le dernier seuil — une arche monumentale gravée de symboles qu'elle commençait à reconnaître, des équations mathématiques peut-être, ou des prières dans une langue que personne ne parlait plus depuis des milliards d'années.

La salle qui s'ouvrait devant eux était immense. Non — le mot était insuffisant. Elle était \textit{impossible}. Une cavité sphérique qui semblait s'étendre sur des kilomètres, bien plus vaste que l'Objet lui-même n'aurait dû le permettre. L'espace replié sur lui-même, avait dit ARIA. La preuve en était là, sous leurs yeux ébahis.

Et partout, suspendues dans le vide lumineux, des capsules.

Des milliers de capsules. Des centaines de milliers. Elles flottaient en formations géométriques parfaites, chacune de la taille d'un cercueil humain, chacune faite de la même matière noire que le reste de l'Objet. À travers leurs parois translucides, on distinguait des formes — des silhouettes, des corps, des restes.

«\,Ce sont...\,», commença Kowalski.

«\,Des sarcophages\,», acheva Élise. «\,Des sarcophages funéraires.\,»

Elle s'avança dans la cavité, la gravité artificielle maintenant ses pieds sur une passerelle invisible qui serpentait entre les capsules. Chaque pas la rapprochait de la vérité qu'elle redoutait de découvrir.

La première capsule qu'elle examina de près contenait un être qu'elle reconnut : l'un des constructeurs, ces créatures élancées aux doigts multiples dont les hologrammes leur avaient parlé. Il était couché dans une position qui évoquait le sommeil, les mains croisées sur la poitrine, les yeux clos pour l'éternité.

«\,Ils se sont enterrés ici\,», murmura-t-elle. «\,Eux-mêmes. Leur propre espèce.\,»

Mais les capsules suivantes racontaient une histoire différente. Les corps qu'elles contenaient n'étaient pas ceux des constructeurs — c'étaient d'autres êtres, d'autres formes de vie. Des créatures tentaculaires aux corps asymétriques ; des humanoïdes trapus recouverts d'écailles ; des entités cristallines dont la structure défiant toute biologie connue. Chaque capsule contenait une espèce différente, un représentant d'une civilisation morte.

«\,Ce ne sont pas les leurs\,», dit Kowalski, qui l'avait suivie. «\,Ces corps... ce sont les espèces qu'ils ont trouvées pendant leur voyage.\,»

«\,Les civilisations mortes des hologrammes\,», confirma Élise. «\,Ils les ont collectées. Préservées.\,»

Elle comprenait maintenant. L'Objet n'était pas seulement le tombeau des constructeurs — c'était un mausolée universel. Un monument à toutes les formes de vie intelligente que l'univers avait produites puis détruites. Une archive de la conscience, conservée pour l'éternité dans le froid absolu de la Ceinture de Kuiper.

«\,Combien\,?\,» demanda-t-elle à ARIA, sa voix à peine plus qu'un souffle.

«\,D'après mes estimations basées sur la densité des capsules et le volume apparent de la cavité... approximativement quatre-vingt-sept mille espèces distinctes.\,»

Quatre-vingt-sept mille. Quatre-vingt-sept mille civilisations qui avaient existé, qui avaient pensé, qui avaient regardé les étoiles et s'étaient demandé si elles étaient seules. Et qui étaient mortes, toutes, sans exception, avant que l'humanité n'existe.

Élise sentit ses genoux fléchir. Elle s'appuya contre une passerelle, le souffle court, submergée par l'immensité de ce qu'elle contemplait.

«\,Nous sommes les derniers\,», murmura-t-elle. «\,Nous sommes vraiment les derniers.\,»

Non — pas les derniers. Les seuls. Tous les autres s'étaient éteints bien avant que la Terre ne se forme. L'humanité n'était pas l'héritière d'un univers peuplé ; elle était la première — et peut-être la seule — espèce consciente à contempler un cosmos vide.

«\,Élise.\,»

La voix de Kowalski la tira de sa stupeur. Elle leva les yeux et vit qu'il pointait quelque chose au centre de la cavité — une structure plus massive que les autres, entourée d'un halo de lumière bleutée.

«\,Il y a quelque chose là-bas. Une plate-forme centrale.\,»

Élise se redressa, rassemblant ce qui lui restait de force. Ils n'étaient pas venus jusqu'ici pour s'effondrer. Ils étaient venus pour comprendre.

La passerelle les mena jusqu'au centre de la cavité. La plate-forme qui s'y trouvait était circulaire, une dizaine de mètres de diamètre, et en son milieu se dressait une structure cristalline qui pulsait d'une lumière propre.

«\,Un terminal\,», dit Élise en s'approchant. «\,Comme les consoles que nous avons vues plus tôt, mais plus grand. Plus... important.\,»

Elle tendit la main vers le cristal, et celui-ci s'illumina.

Les hologrammes réapparurent — non pas les constructeurs cette fois, mais quelque chose de différent. Des images abstraites, des schémas, des représentations de galaxies et d'amas stellaires. Et au centre de tout, une figure lumineuse qui semblait les regarder.

\textit{Vous avez trouvé notre repos}, dit la présence dans leur esprit. \textit{Vous avez vu ce que nous avons vu. Maintenant, écoutez ce que nous avons compris.}

La transmission commença.
