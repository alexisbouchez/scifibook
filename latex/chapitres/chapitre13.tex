% ============================================================================
%                    CHAPITRE 13 — LES HOLOGRAMMES
% ============================================================================

\chapter{Les Hologrammes}

\bigskip

Ils les virent pour la première fois dans une salle circulaire, plus profondément enfouie dans les entrailles de l'Objet.

Élise avait cessé de compter les corridors qu'ils avaient empruntés, les escaliers impossibles qu'ils avaient gravis, les espaces aux géométries déformées qu'ils avaient traversés. Le temps lui-même semblait s'être dilaté ; leurs chronographes indiquaient quatre heures depuis leur entrée, mais son corps en ressentait bien davantage.

Puis ils débouchèrent dans cette salle, et le temps s'arrêta tout à fait.

Des silhouettes se dressaient au centre de l'espace — translucides, lumineuses, figées dans des poses qui évoquaient la contemplation ou la prière. Elles n'étaient pas humaines, cela sautait aux yeux ; mais elles n'en étaient pas si éloignées qu'on ne pût les reconnaître pour ce qu'elles étaient : des êtres pensants, des créatures conscientes, des \textit{personnes}.

«\,Mon Dieu\,», souffla Kowalski. «\,Ce sont... ce sont eux. Les constructeurs.\,»

Élise s'avança, fascinée. Les silhouettes mesuraient environ deux mètres de haut — plus grandes que la moyenne humaine, plus élancées aussi. Leurs membres supérieurs se terminaient par ce qui ressemblait à des mains, bien que les doigts fussent plus longs et plus nombreux. Leurs têtes, ovales et lisses, étaient dépourvues de traits distinctifs, mais quelque chose dans leur posture suggérait une intelligence profonde, une sagesse accumulée sur des millions d'années.

Elle tendit la main vers la silhouette la plus proche, et quelque chose se produisit.

L'hologramme s'anima.

Les yeux — car c'en était bien, elle le comprit maintenant, deux ovales plus sombres sur la surface de la tête — s'ouvrirent lentement. La créature tourna son regard vers Élise, et pendant un instant vertigineux, deux intelligences se contemplèrent à travers le gouffre des milliards d'années.

Puis l'hologramme parla.

Ce n'était pas une voix, pas vraiment — plutôt une sensation directement implantée dans son esprit, des concepts qui se formaient sans passer par le langage. Des images, des émotions, des fragments de pensée qui n'appartenaient pas à l'humanité.

\textit{Bienvenue}, disait le message. \textit{Nous avons attendu. Nous avons espéré. Vous êtes venus.}

Élise chancela, submergée par le flot d'informations. À côté d'elle, Kowalski s'était agenouillé, une main pressée contre son casque, le visage déformé par l'effort de comprendre.

«\,Vous... vous entendez ça\,?\,» demanda-t-il d'une voix rauque.

«\,Oui.\,»

L'hologramme continuait, indifférent à leur trouble. Les images affluaient — des planètes vues depuis l'espace, des villes aux architectures impossibles, des créatures aux formes variées qui travaillaient, vivaient, mouraient. L'histoire d'une civilisation qui s'était étendue aux étoiles, qui avait construit des merveilles, qui avait cherché pendant des millions d'années d'autres esprits avec qui partager l'immensité de l'univers.

Et qui n'en avait jamais trouvé.

\textit{Nous avons parcouru la galaxie}, disait le message. \textit{Nous avons envoyé des sondes vers les amas lointains, nous avons écouté pendant des éons. Partout, le silence. Partout, la mort. Des civilisations qui avaient été, qui n'étaient plus. Des mondes où la vie avait brûlé puis s'était éteinte. Nous étions seuls.}

Les images devinrent plus sombres. Des soleils qui mouraient, des planètes qui se refroidissaient, des cités qui tombaient en ruine. Le déclin d'une espèce qui avait tout exploré et ne trouvait plus de raison de continuer.

\textit{Nous savions que notre fin approchait. Nous voulions laisser quelque chose — un témoin, un message, une preuve que nous avions existé. Nous avons construit ceci.}

L'hologramme désigna les murs autour d'eux, l'Objet tout entier.

\textit{Et nous avons attendu. Attendu que quelqu'un vienne. Que quelqu'un entende. Que quelqu'un comprenne.}

Le message s'interrompit, et l'hologramme se figea de nouveau — statue de lumière dans un musée de ténèbres. Élise resta immobile, le cœur battant, l'esprit débordant de tout ce qu'elle venait de recevoir.

«\,Thulé, vous recevez\,?\,» demanda-t-elle d'une voix qui tremblait.

Un grésillement, puis la voix d'Okonkwo, lointaine et déformée.

«\,On reçoit... mal. Qu'est-ce qui se passe là-bas\,?\,»

«\,Nous avons trouvé... nous avons trouvé les créateurs. Ou ce qu'il en reste.\,» Elle prit une inspiration, essayant d'ordonner ses pensées. «\,Ce sont des enregistrements. Des hologrammes interactifs. Ils nous ont laissé un message.\,»

«\,Qu'est-ce qu'il dit\,?\,»

Élise regarda les silhouettes figées autour d'elle — ces êtres qui avaient parcouru la galaxie, qui avaient bâti des empires parmi les étoiles, qui avaient fini par mourir seuls, comme tous les autres.

«\,Ils nous ont dit qu'ils étaient seuls\,», répondit-elle. «\,Et que nous le sommes peut-être aussi.\,»
