% ============================================================================
%                      CHAPITRE 19 — L'ÉVEIL
% ============================================================================

\chapter{L'Éveil}

\bigskip

Elle ouvrit les yeux sur le plafond blanc de l'infirmerie du \textit{Thulé}.

Combien de temps s'était-il écoulé\,? Elle n'aurait su le dire. La lumière des néons lui brûlait les rétines ; le bourdonnement des systèmes de vie lui emplissait les oreilles ; son corps entier lui semblait étranger, comme si elle l'habitait pour la première fois.

«\,Elle revient à elle.\,»

La voix d'Okonkwo, quelque part sur sa droite. Puis un visage au-dessus du sien — Vasquez, le front plissé d'inquiétude malgré son expression sévère.

«\,Morneau. Vous m'entendez\,?\,»

«\,Je...\,» Sa voix était rauque, ses lèvres craquelées. «\,Oui. Je vous entends.\,»

«\,Vous nous avez fait une sacrée peur. Quatre heures inconsciente. Vos constantes vitales ont fait des montagnes russes.\,» Vasquez secoua la tête. «\,Je devrais vous faire passer en cour martiale pour insubordination.\,»

«\,Vous ne le ferez pas.\,»

«\,Non. Mais j'y ai pensé.\,»

Élise essaya de se redresser. Ses muscles protestèrent, raides comme après une longue maladie, mais elle parvint à s'asseoir sur le bord de la couchette. Okonkwo lui tendit un verre d'eau qu'elle but avidement.

«\,Qu'est-ce qui s'est passé\,?\,» demanda la jeune exobiologiste. «\,Quand vous avez touché le cristal, les instruments sont devenus fous. ARIA a détecté une activité neuronale sans précédent, puis plus rien. On a cru qu'on vous avait perdue.\,»

Élise regarda ses mains — ces mains qui avaient touché l'infini, qui avaient reçu la sagesse d'une civilisation morte depuis des milliards d'années. Elles tremblaient légèrement.

«\,J'ai parlé avec eux\,», dit-elle. «\,Les constructeurs. Pas leurs enregistrements — eux. Ce qu'il en reste.\,»

Vasquez et Okonkwo échangèrent un regard.

«\,Et qu'est-ce qu'ils vous ont dit\,?\,»

Comment résumer\,? Comment condenser cette expérience — cette communion avec une intelligence si vaste, si ancienne — en quelques phrases compréhensibles\,? Les mots lui semblaient dérisoires, inadéquats.

«\,Ils m'ont montré... tout. L'histoire de la galaxie. Les civilisations qui ont existé et disparu. La raison pour laquelle ils ont construit l'Objet.\,» Elle ferma les yeux un instant, cherchant à organiser le chaos de ses souvenirs. «\,Ils voulaient que quelqu'un sache. Que quelqu'un comprenne ce qu'ils avaient compris.\,»

«\,Et qu'est-ce qu'ils avaient compris\,?\,»

Élise rouvrit les yeux et regarda Okonkwo — cette jeune femme brillante qui avait cru si fort à l'existence d'autres formes de vie, qui avait été dévastée par la découverte du mausolée.

«\,Que nous sommes précieux\,», dit-elle. «\,Infiniment précieux. Pas malgré notre solitude — à cause d'elle. La conscience est rare, Amara. Plus rare que tout ce que nous imaginions. Et chaque instant où un esprit regarde l'univers et se pose des questions... c'est un miracle.\,»

Le silence qui suivit fut long, chargé. Quelque part dans les entrailles du vaisseau, un système de ventilation soufflait doucement.

«\,C'est tout\,?\,» demanda Vasquez. «\,Pas de technologie révolutionnaire\,? Pas de secret de l'immortalité\,? Pas de plans pour des moteurs supraluminiques\,?\,»

Élise eut un sourire fatigué.

«\,Non. Ils ne nous ont pas donné d'outils, Commandante. Ils nous ont donné une perspective.\,» Elle marqua une pause. «\,Leur technologie n'aurait servi à rien de toute façon. Elle était conçue pour leur façon de penser, pas la nôtre. Ce qu'ils ont vraiment légué, c'est une compréhension. De ce que nous sommes. De ce que nous valons.\,»

«\,Et cette compréhension valait le risque que vous avez pris\,?\,»

Élise considéra la question. Avait-elle eu raison de connecter son esprit au cristal, de risquer sa vie — sa promesse à Léa — pour obtenir cette connaissance\,?

«\,Je ne sais pas\,», admit-elle. «\,Mais je sais que je devais essayer. Et maintenant que je sais ce qu'ils voulaient nous dire...\,» Elle regarda vers le hublot, vers l'espace noir parsemé d'étoiles. «\,Maintenant je dois décider comment le transmettre.\,»

Vasquez hocha la tête.

«\,Reposez-vous d'abord. Nous avons huit mois de voyage devant nous. Vous aurez tout le temps de décider.\,»

Huit mois. Huit mois pour digérer ce qu'elle avait vécu, pour formuler ce qu'elle avait appris, pour choisir les mots qui porteraient le message des constructeurs à une humanité qui ne se doutait de rien.

Élise se rallongea sur la couchette, épuisée. Mais avant de fermer les yeux, elle murmura une dernière chose :

«\,Il faut qu'on parte. Il n'y a plus rien à chercher ici.\,»

Vasquez acquiesça.

«\,Je donne l'ordre de départ. Direction : la Terre.\,»
