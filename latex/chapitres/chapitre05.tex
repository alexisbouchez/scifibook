% ============================================================================
%                       CHAPITRE 5 — LE CHOIX
% ============================================================================

\chapter{Le Choix}

\bigskip

Deux mois s'étaient écoulés depuis la réunion de l'ESA, et le monde avait basculé.

L'annonce officielle, soigneusement orchestrée par les services de communication des principales agences spatiales, avait provoqué exactement les réactions qu'Élise avait anticipées : stupeur, incrédulité, puis cette fièvre étrange qui s'empare des foules lorsqu'elles comprennent que l'impossible vient de se produire. Les chaînes d'information diffusaient en boucle les images de la sphère noire ; les réseaux sociaux débordaient de théories, de prières, de peurs et d'espoirs ; les gouvernements tenaient des réunions de crise ; et quelque part, dans les chantiers navals orbitaux de l'ESA, le vaisseau \textit{Thulé} prenait forme à une vitesse que personne n'aurait crue possible quelques mois plus tôt.

Élise, elle, se tenait dans la cuisine de son appartement parisien, face à sa fille.

Léa avait seize ans, les cheveux noirs de son père et les yeux gris de sa mère, et ce regard — ce regard qu'Élise connaissait trop bien — qui disait sans un mot : \textit{Je sais ce que tu vas m'annoncer. Je le sais depuis des semaines.}

«\,J'ai été sélectionnée\,», dit Élise.

Les mots tombèrent dans le silence de l'appartement comme des pierres dans l'eau.

«\,Je sais\,», répondit Léa.

Elle était assise sur un tabouret, les coudes posés sur le comptoir de granit, une tasse de thé refroidi devant elle. Elle ne pleurait pas. Elle ne criait pas. Elle se contentait de regarder sa mère avec cette expression de résignation qui faisait plus mal à Élise que n'importe quelle colère.

«\,La mission durera dix-huit mois\,», poursuivit Élise, consciente qu'elle récitait des informations que Léa connaissait déjà. «\,Huit mois pour l'aller, quelques semaines sur place, huit mois pour le retour. Je serai de retour avant tes dix-huit ans.\,»

«\,Peut-être.\,»

Le mot flotta entre elles, chargé de tout ce qu'il impliquait. \textit{Peut-être.} Parce que les missions spatiales lointaines comportaient des risques que l'on ne pouvait jamais totalement éliminer. Parce que personne ne savait ce qu'ils trouveraient là-bas. Parce que Élise pouvait très bien ne jamais revenir.

«\,Léa...\,»

«\,Quoi\,?\,» Sa fille leva les yeux, et Élise y vit quelque chose qu'elle n'avait pas vu depuis longtemps — non pas de la colère, mais une sorte de lassitude ancienne, celle des enfants qui ont appris trop tôt que l'amour de leurs parents a des limites. «\,Qu'est-ce que tu veux que je te dise, maman\,? Que je suis heureuse pour toi\,? Que je comprends\,? Que c'est formidable, la plus grande découverte de l'histoire, et que tu dois y aller parce que c'est ton destin, parce que tu as travaillé toute ta vie pour ce moment\,?\,»

«\,Ce n'est pas—\,»

«\,Si.\,» Léa secoua la tête, un sourire amer au coin des lèvres. «\,C'est exactement ça. C'est toujours ça. Quand papa est tombé malade, tu passais tes nuits au laboratoire. Quand il est mort, tu t'es réfugiée dans ton travail. Chaque anniversaire, chaque Noël, chaque moment où j'avais besoin de toi, il y avait toujours quelque chose de plus important. Une découverte, une publication, une conférence. Et maintenant...\,» Elle désigna le plafond d'un geste vague, comme si l'Objet de Thulé flottait juste au-dessus d'eux. «\,Maintenant, il y a ça. La plus grande découverte de tous les temps. Comment est-ce que je pourrais rivaliser avec ça\,?\,»

Élise sentit les mots se bloquer dans sa gorge. Elle aurait voulu protester, dire que ce n'était pas vrai, que Léa comptait plus que tout le reste. Mais les mots sonnaient faux avant même d'être prononcés, parce que Léa avait raison. Elle avait toujours eu raison.

«\,Je pourrais dire non\,», murmura Élise.

«\,Non, tu ne pourrais pas.\,»

Les deux femmes se regardèrent en silence. Dehors, Paris vivait sa vie ordinaire — le bruit des voitures, les conversations des passants, le monde qui continuait de tourner sans se soucier du drame qui se jouait dans cette cuisine. Et quelque part, à des milliards de kilomètres de là, une sphère noire attendait.

«\,Ta tante Hélène a accepté que tu viennes vivre chez elle\,», dit enfin Élise. «\,À Lyon. Tu pourras continuer tes études au même lycée, j'ai vérifié les équivalences. Et je t'appellerai tous les jours, le décalage temporel permet—\,»

«\,Maman.\,» Léa leva la main pour l'interrompre. «\,Tu n'as pas besoin de te justifier. Je sais que tu vas partir. Je l'ai su dès l'instant où j'ai vu ton nom dans la liste des candidats potentiels. Tu reviens toujours aux étoiles. C'est ce que tu es. C'est ce que tu as toujours été.\,»

\textit{Tu reviens toujours aux étoiles.}

Les mots résonnèrent dans l'esprit d'Élise comme le verdict d'un procès qu'elle avait perdu depuis longtemps. Elle pensa à Marc, à ces derniers mois où elle avait partagé son temps entre l'hôpital et l'Observatoire, incapable de choisir entre l'homme qu'elle aimait et l'univers qui l'appelait. Elle pensa aux anniversaires manqués, aux spectacles d'école auxquels elle n'avait pas assisté, aux milliers de petits renoncements qui avaient, goutte après goutte, creusé un abîme entre elle et sa fille.

Et elle pensa à l'Objet. À cette sphère impossible qui flottait aux confins du système solaire, chargée de secrets que personne n'avait jamais déchiffrés. À ce message en nombres premiers qui traversait le vide depuis des temps immémoriaux, attendant que quelqu'un vienne enfin l'écouter.

Le choix, elle le savait, était déjà fait. Il avait été fait depuis le premier instant, depuis cette nuit de janvier où elle avait vu les données pour la première fois. Peut-être même avait-il été fait bien plus tôt — le jour où elle avait levé les yeux vers le ciel nocturne, enfant, et compris que sa vie ne serait jamais ailleurs que là-haut.

«\,Je suis désolée\,», dit-elle, et les mots lui parurent si dérisoires qu'elle faillit en rire. «\,Je suis tellement désolée, Léa.\,»

Sa fille se leva, contourna le comptoir, et vint se placer devant elle. Elle était presque aussi grande qu'Élise maintenant, et dans ses yeux gris brillait quelque chose qui ressemblait — contre toute attente — à de la compassion.

«\,Je sais\,», dit-elle doucement. «\,Je sais que tu es désolée. Et je sais que ça ne changera rien.\,»

Elle prit la main de sa mère, la serra brièvement, puis la relâcha.

«\,Reviens\,», ajouta-t-elle. «\,C'est tout ce que je te demande. Peu importe ce que tu trouves là-bas, peu importe ce que ça signifie pour l'humanité ou pour la science. Reviens.\,»

Élise hocha la tête, incapable de parler. Elle attira sa fille contre elle, la serra dans ses bras pour la première fois depuis des mois, et sentit les larmes qu'elle avait retenues si longtemps couler enfin sur ses joues.

Elle partirait. Elle irait voir ce que l'univers avait caché là-bas, aux frontières du connu. Mais une partie d'elle resterait ici, dans cette cuisine, dans ces bras qu'elle allait abandonner.

Une partie d'elle resterait toujours.
