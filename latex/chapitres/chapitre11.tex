% ============================================================================
%                   CHAPITRE 11 — L'OUVERTURE
% ============================================================================

\chapter{L'Ouverture}

\bigskip

Kowalski tint parole.

Quarante-huit heures après l'incident, il était de nouveau dehors, flottant à quelques mètres de la surface noire. Sa jambe blessée le faisait boiter même en apesanteur, mais il avait insisté pour accompagner Élise. «\,Je veux voir\,», avait-il dit simplement. «\,J'ai failli mourir pour ça. Je veux savoir pourquoi.\,»

Vasquez avait cédé, à contrecœur. Elle avait imposé des conditions strictes — communication permanente, rappel immédiat au moindre signe de danger, Okonkwo en veille médicale — mais elle avait cédé. Peut-être parce qu'elle comprenait, au fond, que refuser était impossible. Ils avaient traversé le système solaire pour ce moment. Reculer maintenant aurait été une trahison de tout ce qui les avait amenés ici.

Le motif géométrique brillait faiblement dans la lumière de leurs projecteurs. Le cercle qui s'était entrouvert deux jours plus tôt était de nouveau scellé, la surface revenue à sa perfection originelle comme si rien ne s'était passé.

«\,ARIA\,», dit Élise, «\,analyse du signal original. La séquence des nombres premiers — est-ce qu'elle correspond à quelque chose dans le motif\,?\,»

«\,Analyse en cours.\,» Un silence. «\,Correspondance trouvée. La séquence des nombres premiers, convertie en coordonnées polaires, trace exactement la spirale récurrente du motif.\,»

Élise sourit. Bien sûr. Le signal n'était pas seulement une preuve d'intelligence — c'était une clé. Les créateurs de l'Objet avaient diffusé le code d'accès à travers l'espace, attendant que quelqu'un soit assez avancé pour le comprendre et l'utiliser.

«\,Je vais projeter la séquence complète\,», annonça-t-elle. «\,Tous les nombres premiers, de deux jusqu'à... ARIA, jusqu'où allait le signal original\,?\,»

«\,Jusqu'à mille neuf cent quatre-vingt-dix-sept. Le trois centième nombre premier.\,»

«\,Parfait. Kowalski, tenez-vous prêt. Si quelque chose se passe, on recule immédiatement.\,»

«\,Compris.\,»

Élise programma le projecteur et l'activa. Cette fois, au lieu de la simple spirale, la lumière traçait la séquence entière : 2, 3, 5, 7, 11, 13... Les nombres s'inscrivaient en spirale sur la surface noire, chacun une impulsion lumineuse qui semblait être absorbée par la matière plutôt que réfléchie.

Quand le dernier nombre — 1997 — s'éteignit, le silence retomba.

Puis l'Objet s'ouvrit.

Pas comme une porte qui pivote ou un sas qui coulisse. La surface elle-même se transforma, les lignes du motif s'écartant, se réorganisant, créant une ouverture circulaire d'environ cinq mètres de diamètre. À l'intérieur, il n'y avait pas l'obscurité qu'Élise s'attendait à voir, mais une lumière — douce, bleutée, comme une aube perpétuelle.

«\,Mon Dieu\,», souffla Kowalski.

Élise s'approcha du bord de l'ouverture. L'intérieur se révélait progressivement : des parois lisses, de la même matière noire que l'extérieur mais parcourues de veines lumineuses ; un couloir — car c'était bien un couloir — qui s'enfonçait vers le centre de la sphère selon une courbe impossible.

«\,ARIA, analyse atmosphérique\,», demanda-t-elle.

«\,Aucune atmosphère détectée. L'intérieur est sous vide, comme l'extérieur. Cependant, je détecte un champ gravitationnel artificiel. Approximativement 0,3 g, orienté vers la paroi intérieure.\,»

Une gravité artificielle. Quelqu'un avait conçu cet espace pour qu'on puisse y marcher.

«\,Thulé, vous recevez\,?\,» demanda Élise dans son communicateur.

«\,On reçoit\,», répondit Vasquez. «\,On voit tout sur les caméras. C'est... c'est incroyable.\,»

«\,Demande l'autorisation d'entrer.\,»

Un silence. Élise imaginait Vasquez en train de peser les risques, de calculer les probabilités, de lutter contre tous ses instincts de commandante qui lui hurlaient de rappeler son équipage.

«\,Autorisation accordée\,», dit enfin Vasquez. «\,Mais vous restez en contact permanent. Au moindre problème, vous faites demi-tour.\,»

«\,Compris.\,»

Élise franchit le seuil.

La sensation fut immédiate : ses pieds se posèrent sur la paroi courbe, attirés par le champ gravitationnel artificiel. Ce qui avait été le sol devint le sol, et elle se retrouva debout dans un corridor circulaire, les veines lumineuses pulsant doucement autour d'elle comme les artères d'un organisme vivant.

Kowalski la rejoignit, boitant légèrement.

«\,Incroyable\,», dit-il. «\,La gravité est parfaitement stable. Comment font-ils\,?\,»

«\,Je ne sais pas.\,» Élise avança de quelques pas, touchant la paroi du bout des doigts. Elle était tiède — pas froide comme l'extérieur, mais tiède, comme si une source de chaleur existait quelque part dans les profondeurs de l'Objet. «\,Mais j'ai l'intention de le découvrir.\,»

Ils progressèrent le long du corridor, guidés par la lumière bleutée. Les parois s'élargissaient progressivement, le passage devenant plus haut, plus large, comme s'il les menait vers un espace central. Et partout, ces veines lumineuses qui pulsaient à un rythme régulier — le battement de cœur d'une machine vieille de quatre milliards d'années.

«\,Vous sentez ça\,?\,» demanda Kowalski.

Élise s'arrêta. Oui, elle le sentait. Une vibration dans l'air — non, pas dans l'air, il n'y en avait pas — dans la structure elle-même. Un son grave, à la limite de l'audible, qui semblait émaner de partout et de nulle part.

«\,On dirait que ça... respire\,», murmura-t-elle.

L'Objet de Thulé était vivant.

Ou du moins, quelque chose à l'intérieur l'était encore.
