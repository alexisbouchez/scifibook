% ============================================================================
%                     CHAPITRE 21 — LE DILEMME
% ============================================================================

\chapter{Le Dilemme}

\bigskip

À trois semaines de l'arrivée, l'ESA demanda un rapport préliminaire.

Élise contempla l'écran de communication, le curseur clignotant dans l'attente de sa réponse. De l'autre côté de ces quelques milliards de kilomètres, des centaines de personnes attendaient — scientifiques, politiques, journalistes. Le monde entier retenait son souffle depuis dix-huit mois, impatient de savoir ce que l'équipage du \textit{Thulé} avait découvert.

Que leur dire\,?

La vérité brute était simple à formuler : nous avons trouvé un monument funéraire contenant les restes de quatre-vingt-sept mille civilisations éteintes. Les créateurs de l'Objet nous ont transmis un message : nous sommes probablement seuls dans l'univers, et nous le serons toujours.

Mais cette formulation, Élise le savait, serait dévastatrice. Elle imaginait les titres des journaux, les réactions des foules, le désespoir qui s'emparerait de millions de personnes. Tout espoir de rencontrer d'autres formes de vie, anéanti en une phrase. Toute illusion d'un univers peuplé, brisée à jamais.

Et pourtant, ce n'était pas le vrai message. Pas entièrement.

Les constructeurs n'avaient pas voulu transmettre du désespoir. Ils avaient voulu transmettre de la responsabilité. De la valeur. Du sens. «\,Vous êtes précieux\,», avaient-ils dit. «\,Infiniment précieux.\,» Mais comment faire passer cette nuance à travers le prisme déformant des médias, des interprétations, des peurs humaines\,?

Elle appela Léa ce soir-là — le premier appel vidéo depuis des mois, les communications à longue distance étant limitées pendant le voyage. Le visage de sa fille apparut sur l'écran, plus vieux qu'elle ne s'y attendait, plus adulte.

«\,Maman.\,»

«\,Léa.\,» Élise sentit sa gorge se serrer. «\,Tu vas bien\,?\,»

«\,Ça va. Tante Hélène s'occupe de moi. Je...\,» Léa hésita. «\,Tu reviens bientôt\,?\,»

«\,Trois semaines. Peut-être moins.\,»

Un silence. Les secondes de décalage dues à la distance rendaient la conversation étrange, hachée, comme si les mots devaient traverser un océan avant d'arriver à destination.

«\,Qu'est-ce que vous avez trouvé là-bas\,?\,» demanda Léa. «\,Les journaux disent que vous ne communiquez pas, que c'est le black-out total. Tout le monde spécule.\,»

«\,Nous avons trouvé... beaucoup de choses. Des réponses. Des questions aussi.\,» Élise soupira. «\,C'est compliqué, Léa. Je ne sais pas encore comment l'expliquer.\,»

«\,Tu me le diras quand tu rentreras\,?\,»

«\,Oui. Je te dirai tout.\,»

Le visage de Léa s'adoucit légèrement — une fissure dans le mur qu'elle avait érigé entre elles.

«\,Tu as l'air différente, maman. Je ne sais pas comment l'expliquer. Mais tu as l'air... je ne sais pas. Plus calme, peut-être.\,»

«\,J'ai appris des choses. Sur l'univers. Sur nous.\,» Élise esquissa un sourire. «\,Sur ce qui compte vraiment.\,»

«\,Et qu'est-ce qui compte vraiment\,?\,»

La question resta suspendue dans l'espace entre elles. Élise regarda sa fille — cette jeune femme qu'elle avait si souvent délaissée pour sa carrière, qu'elle avait abandonnée pour poursuivre les étoiles — et sentit quelque chose se nouer dans sa poitrine.

«\,Toi\,», dit-elle. «\,Tu comptes. Chaque instant où nous sommes ensemble, chaque conversation, chaque regard. C'est ça qui compte. J'aurais dû le comprendre plus tôt.\,»

Léa ne répondit pas, mais Élise vit quelque chose briller dans ses yeux — pas des larmes, pas encore, mais quelque chose qui y ressemblait.

«\,Rentre vite, maman.\,»

«\,Je rentre. Je te le promets.\,»

Après avoir coupé la communication, Élise resta longtemps devant l'écran éteint. Puis elle ouvrit un nouveau fichier et commença à écrire.

Pas un rapport. Pas une synthèse scientifique.

Un discours.

Le discours qu'elle prononcerait devant le monde entier, quand viendrait le moment de partager ce que les constructeurs lui avaient confié. Elle ne savait pas encore quels mots elle choisirait, quelle formulation serait la plus juste. Mais elle savait une chose : elle dirait la vérité.

Toute la vérité. Sans adoucissement, sans mensonge réconfortant.

Parce que l'humanité méritait de savoir. Et parce que la vérité, aussi difficile fût-elle, était le seul héritage digne des créateurs qui avaient attendu quatre milliards d'années pour la transmettre.
