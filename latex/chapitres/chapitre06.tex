% ============================================================================
%                       CHAPITRE 6 — LE DÉPART
% ============================================================================

\chapter{Le Départ}

\bigskip

Le jour du lancement, le ciel au-dessus de Kourou était d'un bleu si pur qu'il semblait peint.

Élise se tenait dans le vestibule du centre de préparation, vêtue de la combinaison de vol qui serait son uniforme pour les dix-huit mois à venir. À travers la baie vitrée, elle apercevait le \textit{Thulé} sur son pas de tir — non pas la fusée élancée des lancements traditionnels, mais une structure plus massive, plus complexe, qui ressemblait davantage à une cathédrale de métal qu'à un véhicule spatial. Le vaisseau avait été assemblé en orbite, et seul le module de transport qui les conduirait jusqu'à lui reposait sur la rampe, prêt à s'arracher à la gravité terrestre.

Autour d'elle, l'équipage effectuait les derniers préparatifs. Ils étaient quatre : le nombre minimum pour une mission de cette durée, le maximum que les systèmes de support de vie pouvaient maintenir.

La commandante Sofia Vasquez, quarante-huit ans, ancienne pilote de chasse devenue astronaute, dirigeait les opérations avec cette efficacité tranquille qui caractérisait les vétérans de l'espace. C'était sa cinquième mission, et la première qu'elle avait hésité à accepter. «\,Je préfère les risques que je connais\,», avait-elle confié à Élise quelques semaines plus tôt. «\,Là-bas, on ne sait pas ce qu'on va trouver.\,» Mais elle avait accepté quand même, parce que refuser une telle mission aurait été impensable.

Le Docteur Amara Okonkwo, trente-cinq ans, exobiologiste nigériane formée à Cambridge, était la benjamine de l'équipe. Elle irradiait un optimisme presque naïf qui contrastait avec le sérieux des autres membres. «\,Nous allons rencontrer quelqu'un\,», répétait-elle avec une conviction inébranlable. «\,Ou au moins, les traces de quelqu'un. L'univers ne peut pas être aussi vide qu'il en a l'air.\,» Élise enviait cette foi, même si elle ne la partageait pas.

Piotr Kowalski, quarante-deux ans, ingénieur polonais et spécialiste des activités extravéhiculaires, était le plus silencieux des quatre. Vingt ans d'expérience dans l'espace l'avaient rendu taciturne, presque monolithique ; mais quand il parlait, ses mots portaient le poids d'une compétence éprouvée. C'était lui qui les maintiendrait en vie si les systèmes venaient à flancher, lui qui réparerait ce qui pouvait l'être, lui qui trouverait des solutions quand la science atteindrait ses limites.

Et puis il y avait ARIA.

L'intelligence artificielle du \textit{Thulé} n'avait pas de corps, pas de visage, mais sa présence était partout — dans les haut-parleurs qui diffusaient sa voix calme et mesurée, dans les écrans qui affichaient ses analyses, dans les capteurs qui surveillaient chaque paramètre du vaisseau. Elle était le cinquième membre de l'équipage, celui qui ne dormait jamais, celui qui voyait tout et calculait tout avec une précision dont aucun cerveau humain n'était capable.

«\,T moins deux heures\,», annonça ARIA. «\,Tous les systèmes sont nominaux. La fenêtre de lancement est optimale.\,»

Élise hocha la tête, mais son esprit était ailleurs. Elle pensait à Léa, qu'elle avait embrassée la veille au soir avant de prendre l'avion pour la Guyane. Sa fille n'avait pas pleuré. Elle s'était contentée de la serrer fort, très fort, puis de la relâcher avec ce sourire triste qui hantait encore Élise.

\textit{Reviens.}

Le mot résonnait dans sa tête comme une prière.

«\,Docteur Morneau\,?\,»

Elle se retourna. Vasquez se tenait devant elle, le visage indéchiffrable.

«\,C'est l'heure. Les journalistes veulent une dernière déclaration.\,»

Élise acquiesça et suivit la commandante vers la salle de presse. Une forêt de caméras l'attendait — des dizaines d'objectifs braqués sur elle, des centaines de millions de regards invisibles qui la fixaient depuis les quatre coins du monde. Elle s'avança jusqu'au pupitre, ajusta le micro, et contempla un instant les visages des journalistes. Ils voulaient des mots historiques, des phrases qui passeraient à la postérité. Ils voulaient du sens, de la grandeur, de l'émotion.

«\,Dans quelques heures\,», commença-t-elle, «\,nous quitterons la Terre pour aller à la rencontre de l'inconnu. Nous ne savons pas ce que nous trouverons là-bas. Nous ne savons pas si l'Objet de Thulé nous donnera des réponses, ou s'il nous posera de nouvelles questions. Mais nous y allons quand même, parce que c'est ce que fait l'humanité. Nous allons voir. Nous allons comprendre. Ou du moins, nous allons essayer.\,»

Elle marqua une pause, cherchant les mots justes.

«\,Il y a quatre milliards d'années, quelque chose a placé cet objet aux confins de notre système solaire. Quelque chose qui n'était pas humain. Quelque chose qui, peut-être, a disparu depuis longtemps. Mais le message qu'ils ont laissé nous attend encore. Et nous avons le devoir — envers eux, envers nous-mêmes, envers toutes les générations qui viendront après nous — d'aller le lire.\,»

Elle recula du pupitre, signalant la fin de la déclaration. Les questions fusèrent, mais elle ne les entendit pas. Son esprit était déjà là-bas, dans le noir absolu de la Ceinture de Kuiper, face à cette sphère parfaite qui l'attendait depuis plus longtemps que la Terre n'existait.

\bigskip

Le lancement eut lieu à quatorze heures, heure locale.

Élise sentit la poussée des moteurs l'écraser contre son siège tandis que le module s'arrachait à la gravité. Par le hublot, elle vit la Guyane rétrécir, puis l'Amérique du Sud, puis la courbe bleue de l'océan Atlantique. En quelques minutes, la Terre entière tenait dans le cadre de la fenêtre — cette petite bille fragile, suspendue dans le noir, où vivaient huit milliards d'âmes qui n'avaient aucune idée de ce qui les attendait.

Ils s'amarrèrent au \textit{Thulé} quatre heures plus tard. Le vaisseau était immense — cent vingt mètres de long, une masse de plusieurs milliers de tonnes qui semblait incongrument délicate dans le vide de l'espace. Élise flotta jusqu'au poste d'observation principal et s'immobilisa devant le grand hublot qui occupait toute la paroi.

La Terre brillait en contrebas, auréolée de l'éclat doré du soleil couchant. Élise la contempla en silence, consciente que c'était la dernière fois qu'elle la verrait ainsi pendant dix-huit mois. Quelque part là-bas, sur cette sphère de roche et d'eau, Léa dormait peut-être, ou regardait le ciel, ou pensait à elle.

«\,Allumage des moteurs dans dix minutes\,», annonça ARIA. «\,Trajectoire confirmée. Temps de transit estimé : huit mois, trois jours, quatorze heures.\,»

Huit mois. Plus de deux cents jours à traverser le vide, à s'éloigner de tout ce qu'elle avait connu, pour aller à la rencontre de l'impossible.

«\,Allumage confirmé\,», dit Vasquez depuis le poste de pilotage. «\,Cap sur Thulé.\,»

Les moteurs ioniques s'éveillèrent avec un murmure presque imperceptible, et le \textit{Thulé} commença son long voyage vers les ténèbres. Élise resta devant le hublot, regardant la Terre qui rétrécissait lentement — point lumineux qui pâlissait, s'amenuisait, se perdait parmi les étoiles.

Bientôt, elle ne serait plus qu'un souvenir.

Et devant eux, là où la lumière du Soleil ne parvenait plus qu'à peine, quelque chose de très ancien attendait d'être découvert.
