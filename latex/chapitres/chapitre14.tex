% ============================================================================
%                      CHAPITRE 14 — LA FISSURE
% ============================================================================

\chapter{La Fissure}

\bigskip

Okonkwo ne dormit pas cette nuit-là.

Élise l'observait depuis le seuil du module commun, hésitant à intervenir. La jeune exobiologiste était assise dans un coin, les genoux ramenés contre sa poitrine, le regard perdu dans le vide. Elle n'avait pas mangé depuis leur retour de l'Objet, n'avait pas parlé non plus — juste ce silence, ce repli sur soi qui inquiétait tout l'équipage.

«\,Elle a besoin de temps\,», dit Vasquez en apparaissant à côté d'Élise. «\,Ce qu'elle a vu... ce que nous avons tous vu dans les transmissions... ça change tout ce en quoi elle croyait.\,»

«\,Elle pensait qu'on trouverait de la vie\,», murmura Élise. «\,Que l'univers grouillait de civilisations, qu'on n'avait qu'à tendre la main pour les rencontrer.\,»

«\,Et à la place, on a trouvé un cimetière.\,»

Élise hocha la tête. Le mot était juste. L'Objet n'était pas un vaisseau, pas une balise, pas un monument aux vivants. C'était un mémorial aux morts — un dernier cri dans le vide, lancé par une espèce qui avait cherché des compagnons pendant des millions d'années et n'avait trouvé que des tombes.

«\,Nous devons retourner là-bas\,», dit Élise.

Vasquez se raidit.

«\,Non.\,»

«\,Commandante—\,»

«\,J'ai dit non.\,» La voix de Vasquez était dure, inflexible. «\,Regardez-la.\,» Elle désigna Okonkwo. «\,Elle a passé vingt minutes à recevoir ces transmissions depuis le vaisseau, et elle est dans cet état. Vous et Kowalski avez été exposés directement pendant des heures. Comment pouvez-vous être sûre que vous n'êtes pas affectés vous aussi\,?\,»

«\,Je suis parfaitement—\,»

«\,Vous êtes obsédée.\,» Vasquez se tourna vers elle, et Élise vit dans ses yeux quelque chose qu'elle n'avait jamais vu chez la commandante : de la peur. «\,Depuis que nous avons quitté la Terre, vous ne pensez qu'à une chose — entrer dans cet Objet, découvrir ses secrets, comprendre ce qu'il contient. Vous êtes prête à tout risquer pour ça. Votre vie, la nôtre, tout.\,»

«\,C'est pour ça que nous sommes venus.\,»

«\,Non.\,» Vasquez secoua la tête. «\,Nous sommes venus pour explorer, pas pour nous sacrifier. Il y a une différence entre la curiosité et la folie, Docteur Morneau. Et vous êtes en train de franchir cette ligne.\,»

Le silence qui suivit fut lourd de tension. Les deux femmes s'affrontaient du regard — la scientifique et la militaire, la quête de connaissance et l'instinct de survie.

«\,Si je peux me permettre\,», intervint la voix d'ARIA, «\,les données physiologiques du Docteur Morneau et de Monsieur Kowalski ne montrent aucune anomalie. Leur exposition aux hologrammes ne semble pas avoir eu d'effets néfastes mesurables.\,»

«\,Et Okonkwo\,?\,» demanda Vasquez.

«\,Le Docteur Okonkwo présente des signes de stress post-traumatique. Ses constantes vitales sont élevées, son sommeil est perturbé. Cependant, ces symptômes sont compatibles avec un choc émotionnel ordinaire, non avec une pathologie externe.\,»

«\,Ce n'est pas une pathologie\,», murmura Okonkwo depuis son coin.

Tous les regards se tournèrent vers elle. Elle avait relevé la tête, et ses yeux brillaient de larmes qu'elle ne cherchait plus à retenir.

«\,J'ai passé ma vie à croire que l'univers était plein de vie\,», dit-elle. «\,Que quelque part, là-haut, il y avait d'autres consciences, d'autres regards, d'autres esprits qui contemplaient les mêmes étoiles que nous. C'était ma foi. Ma raison de me lever chaque matin. Et maintenant...\,»

Elle fit un geste vers le hublot, vers l'Objet qui flottait quelque part dans le noir.

«\,Maintenant je sais qu'ils ont cherché pendant des millions d'années. Des millions d'années. Ils avaient des technologies que nous ne pouvons même pas imaginer, des sondes qui traversaient les galaxies, des instruments capables de détecter la moindre trace de vie. Et ils n'ont rien trouvé. Rien. Nous sommes seuls.\,»

«\,Nous ne savons pas ça\,», dit Élise.

«\,Vraiment\,?\,» Okonkwo eut un rire amer. «\,Vous avez vu les mêmes images que moi. Des milliers de mondes, tous morts. Toutes les civilisations qu'ils ont rencontrées, disparues. L'univers est un cimetière, Élise. Et nous ne sommes que les derniers fossoyeurs.\,»

Élise s'approcha d'elle, s'agenouilla à sa hauteur.

«\,Ce n'est pas ce qu'ils ont dit\,», dit-elle doucement. «\,Le message n'est pas terminé. Ils ne nous ont pas tout montré.\,»

«\,Comment le savez-vous\,?\,»

«\,Parce que les hologrammes se sont arrêtés. Interrompus. Comme s'il y avait une suite, mais qu'on ne l'avait pas encore déclenchée.\,» Élise prit les mains d'Okonkwo dans les siennes. «\,Ils ont laissé ce message pour nous, Amara. Pour quelqu'un qui viendrait un jour. Pourquoi feraient-ils ça si le seul message était "vous êtes seuls, abandonnez tout espoir"\,? Ça n'a pas de sens.\,»

Okonkwo la regarda, et Élise vit dans ses yeux l'ombre d'un doute — le premier depuis des heures.

«\,Vous croyez qu'il y a autre chose\,?\,»

«\,J'en suis certaine. Et je compte bien aller le découvrir.\,»

Elle se releva, se tourna vers Vasquez.

«\,Demain\,», dit-elle. «\,Une dernière exploration. Jusqu'au centre de l'Objet. Si je me trompe, si tout ce qu'il y a là-dedans c'est de la mort et du désespoir, nous rentrerons. Mais si j'ai raison...\,»

Vasquez la dévisagea pendant un long moment. Puis, lentement, elle hocha la tête.

«\,Demain\,», dit-elle. «\,Mais c'est la dernière fois.\,»
