% ============================================================================
%                      CHAPITRE 7 — L'ARRIVÉE
% ============================================================================

\chapter{L'Arrivée}

\bigskip

Huit mois dans le vide avaient appris à Élise ce que signifiait vraiment le mot \textit{immensité}.

Elle avait cru connaître l'espace. Elle avait passé trente ans à l'étudier, à scruter ses profondeurs à travers des télescopes et des données numériques, à en cartographier les mystères depuis le confort de son bureau terrestre. Mais vivre dans l'espace, traverser l'espace, sentir le poids de ces milliards de kilomètres peser sur sa conscience jour après jour — c'était autre chose entièrement.

Le \textit{Thulé} était devenu leur monde. Cent vingt mètres de couloirs, de laboratoires, de quartiers d'habitation ; un cocon de métal et de plastique dérivant dans un océan de néant. Ils avaient établi des routines — repas à heures fixes, exercices physiques pour lutter contre l'atrophie musculaire, sessions de travail, rares moments de loisir. Ils avaient appris à se supporter, à respecter les silences de chacun, à éviter les frictions qui auraient pu, dans un espace aussi confiné, dégénérer en conflits ouverts.

Et puis, un matin — si l'on pouvait appeler ainsi les cycles artificiels imposés par ARIA —, l'Objet était apparu.

Élise flottait dans le module d'observation lorsque les premiers capteurs l'avaient détecté. Une alerte discrète, suivie de la voix calme d'ARIA :

«\,Contact visuel établi. Objet de Thulé à quatre-vingt mille kilomètres. Acquisition radar confirmée.\,»

Elle s'était précipitée vers le hublot, le cœur battant, et l'avait vu.

Une sphère. Une sphère d'un noir si absolu qu'elle semblait moins flotter dans l'espace que le dévorer. Les étoiles qui auraient dû briller derrière elle avaient disparu, engloutis par cette surface qui ne reflétait rien, n'émettait rien, ne trahissait rien de sa nature.

C'était exactement comme les images de Kepler-III l'avaient montré. Et c'était infiniment plus terrifiants.

Les autres membres de l'équipage l'avaient rejointe, attirés par l'alerte. Vasquez s'immobilisa à côté d'elle, le visage impénétrable ; Okonkwo laissa échapper un souffle qui ressemblait à une prière ; Kowalski se contenta de hocher la tête, comme s'il validait mentalement une liste de paramètres techniques.

«\,Mon Dieu\,», murmura Okonkwo. «\,C'est réel. C'est vraiment réel.\,»

«\,Bien sûr que c'est réel\,», répondit Vasquez. «\,On n'a pas traversé la moitié du système solaire pour une illusion d'optique.\,»

Mais sa voix trahissait une tension qu'Élise n'avait jamais entendue chez elle. La commandante, qui avait affronté des pannes catastrophiques en orbite et des atterrissages d'urgence sur la Lune, regardait l'Objet avec quelque chose qui ressemblait à de l'appréhension.

«\,ARIA\,», dit Élise, «\,analyse spectrale.\,»

«\,En cours. La surface absorbe quatre-vingt-dix-neuf virgule quatre-vingt-dix-sept pour cent de la lumière visible. L'albédo est le plus faible jamais enregistré pour un objet du système solaire. Composition cohérente avec les analyses préliminaires : carbone, silicium, fer, titane, et les trois éléments non identifiés précédemment signalés.\,»

«\,Et le signal\,?\,» demanda Kowalski. «\,Le signal radio\,?\,»

Un silence.

«\,Aucune émission détectée\,», répondit ARIA. «\,Le signal sur 1420 mégahertz a cessé.\,»

Élise fronça les sourcils. Le signal des nombres premiers, qui avait été leur boussole pendant des mois, leur preuve irréfutable d'une intelligence derrière l'Objet, s'était tu. Comme si celui-ci avait senti leur approche et décidé de se taire.

«\,Quand\,?\,» demanda-t-elle.

«\,Le signal a cessé il y a approximativement six heures, lorsque nous sommes entrés dans un rayon de cent mille kilomètres.\,»

Okonkwo frissonna.

«\,Il sait que nous sommes là\,», dit-elle. «\,Il nous a vus venir.\,»

«\,Ne tirons pas de conclusions hâtives\,», tempéra Vasquez. «\,Il peut y avoir des dizaines d'explications. Une programmation automatique, une économie d'énergie, un dysfonctionnement...\,»

«\,Ou une réaction à notre présence\,», compléta Élise.

Les deux femmes échangèrent un regard. Pendant huit mois, elles avaient débattu de ce qu'elles trouveraient, de ce que l'Objet pourrait être. Un vaisseau abandonné, peut-être. Un monument. Une balise. Une arme. Les théories ne manquaient pas, mais aucune ne préparait vraiment à cet instant — cet instant où l'on se retrouvait face à l'inconnu, avec pour seule certitude que rien de ce qu'on avait imaginé ne correspondrait à la réalité.

«\,ARIA\,», dit Vasquez, «\,calcule une trajectoire d'approche. Je veux qu'on se mette en orbite à dix kilomètres de la surface. Prudemment.\,»

«\,Trajectoire calculée. Temps de transit : quatorze heures. Consommation de carburant dans les paramètres acceptables.\,»

Le \textit{Thulé} ajusta son cap, et la sphère noire grossit imperceptiblement dans le hublot. Élise ne pouvait pas détacher son regard de cette surface parfaite, de cette courbe immaculée qui semblait défier les lois mêmes de la physique. Quelque part dans son esprit, une voix lui soufflait qu'elle aurait dû avoir peur — qu'un objet aussi ancien, aussi mystérieux, aussi manifestement \textit{autre} aurait dû lui inspirer de la terreur.

Mais ce qu'elle ressentait n'était pas de la peur. C'était de l'émerveillement.

Après cinquante-deux ans de questions, l'univers s'apprêtait enfin à lui répondre.
