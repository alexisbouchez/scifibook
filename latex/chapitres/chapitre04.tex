% ============================================================================
%                      CHAPITRE 4 — LA PREUVE
% ============================================================================

\chapter{La Preuve}

\bigskip

La salle de conférence du siège de l'ESA, rue Mario-Nikis, avait vu passer bien des annonces au cours de ses cinquante années d'existence. Le lancement d'Ariane 6, la découverte des premières exoplanètes habitables, l'établissement de la base lunaire Europa : autant d'étapes qui avaient, chacune à leur manière, repoussé les frontières de l'humanité. Mais ce qui allait s'y dire aujourd'hui éclipsait tout le reste.

Élise se tenait debout devant l'écran principal, ses notes à la main, face à douze visages dont l'expression oscillait entre le scepticisme poli et l'incrédulité à peine dissimulée. Elle les connaissait presque tous — collègues, rivaux, mentors — et elle savait exactement ce qui se passait derrière leurs fronts soucieux. Ils voulaient croire, mais ils ne pouvaient pas se permettre de croire. Pas encore. Pas sans preuves irréfutables.

«\,Mesdames, Messieurs\,», commença-t-elle, «\,permettez-moi de vous présenter ce que nous avons appelé, à titre provisoire, l'Objet de Thulé.\,»

L'image de la sphère noire s'afficha sur l'écran, et un murmure parcourut l'assemblée. Élise laissa quelques secondes s'écouler — le temps que l'impact visuel fît son effet — avant de poursuivre.

«\,Diamètre : douze kilomètres trois cent quarante-sept mètres. Orbite : parfaitement circulaire, à quarante-sept unités astronomiques du Soleil. Période orbitale : trois cent douze années terrestres. Et voici\,», elle appuya sur une touche, et le son des nombres premiers emplit la pièce, «\,le signal qu'il émet sur 1420 mégahertz, de manière continue, depuis au moins trois semaines.\,»

Le directeur des programmes scientifiques, un homme austère nommé Vandenberg, leva la main.

«\,Comment pouvons-nous être certains qu'il ne s'agit pas d'un artefact instrumental\,? Une contamination terrestre, par exemple\,?\,»

«\,Le signal a été détecté simultanément par ALMA, par le réseau de Parkes en Australie, et par les antennes de Jodrell Bank. Trois installations indépendantes, sur trois continents différents. La corrélation est parfaite.\,»

«\,Et la sphère elle-même\,?\,» intervint une femme aux cheveux blancs — Professeur Nakamura, de l'Institut d'astrophysique de Tokyo, invitée en urgence pour cette réunion. «\,Comment expliquez-vous cette... perfection géométrique\,?\,»

Élise inspira profondément. C'était le moment qu'elle redoutait et espérait tout à la fois — le moment où elle devrait franchir la ligne qui séparait la science de la spéculation, l'observation de l'interprétation.

«\,Je ne l'explique pas\,», dit-elle. «\,Aucun processus naturel connu ne peut produire une sphère parfaite de cette taille. Les astéroïdes ont des formes irrégulières, les planètes naines présentent des aplatissements aux pôles. Seule une force\,» — elle marqua une pause — «\,intentionnelle peut créer une géométrie aussi précise.\,»

Le mot resta suspendu dans l'air, lourd de tout ce qu'il impliquait. \textit{Intentionnelle.} Quelqu'un, quelque part, avait voulu que cet objet existe. Quelqu'un l'avait conçu, fabriqué, placé là où il se trouvait.

«\,Mais ce n'est pas tout\,», continua Élise en faisant apparaître un nouveau graphique. «\,L'analyse spectrographique du rayonnement réfléchi par la surface nous a fourni des données sur sa composition.\,»

Elle désigna une série de pics sur le spectre, des lignes d'absorption que les spectromètres avaient identifiées avec une précision de quelques angströms.

«\,La majorité des éléments sont familiers : carbone, silicium, fer, titane. Mais ici\,» — son doigt pointa trois pics distincts, isolés dans une région du spectre habituellement vide — «\,nous avons quelque chose que nous ne pouvons pas identifier.\,»

Un silence pesant tomba sur l'assemblée.

«\,Que voulez-vous dire par "ne pouvons pas identifier"\,?\,» demanda Vandenberg.

«\,Ces signatures ne correspondent à aucun élément du tableau périodique. Ni à aucun isotope connu. Ni à aucun composé théoriquement prédit par les modèles nucléaires actuels.\,»

«\,C'est impossible\,», souffla quelqu'un au fond de la salle.

«\,Et pourtant.\,» Élise sentit une sorte de calme l'envahir — le calme de ceux qui savent qu'ils disent la vérité, même lorsque la vérité semble inconcevable. «\,Ces trois éléments n'existent pas dans notre compréhension de la physique nucléaire. Ils n'ont jamais été synthétisés dans aucun laboratoire terrestre, jamais observés dans aucun rayonnement stellaire, jamais détectés dans aucune météorite.\,»

Elle laissa les implications se déployer d'elles-mêmes dans les esprits de son auditoire.

«\,Ce que je vous dis\,», conclut-elle d'une voix qui ne tremblait pas, «\,c'est que l'Objet de Thulé est composé, au moins en partie, de matériaux qui ne peuvent pas avoir été produits par les processus naturels de l'univers tel que nous le connaissons. Ce n'est pas un astéroïde. Ce n'est pas une comète. Ce n'est pas une planète naine.\,»

Elle se tourna vers l'écran, vers cette sphère d'un noir absolu qui semblait la regarder en retour.

«\,C'est un artefact. Un objet manufacturé. Et il a été placé là par quelque chose — ou quelqu'un — qui n'est pas humain.\,»

Le silence qui suivit dura peut-être dix secondes, peut-être une éternité. Puis la directrice générale, une Française au regard d'acier nommée Isabelle Mercier, se leva lentement de son siège.

«\,Si ce que vous dites est vrai, Docteur Morneau...\,»

«\,C'est vrai.\,»

«\,... alors nous ne pouvons pas garder cette information secrète indéfiniment. Les gouvernements doivent être informés. L'ONU doit être informée. Et nous devons...\,» Elle s'interrompit, comme si la suite de sa phrase lui échappait. «\,Nous devons envoyer une mission.\,»

Élise hocha la tête. Elle avait passé trois semaines à imaginer ce moment, à anticiper les objections, à préparer ses réponses. Elle avait su, depuis le premier instant, que la seule réponse possible à l'Objet de Thulé était d'aller le voir.

«\,Le vaisseau \textit{Thulé}\,», dit-elle en faisant apparaître les schémas d'un projet qu'elle avait exhumé des archives de l'Agence, «\,a été conçu il y a quinze ans pour une mission vers les objets transneptuniens. Sa construction n'a jamais été achevée, faute de financement. Mais la structure principale existe. Les moteurs ioniques de nouvelle génération peuvent être installés en six mois. Avec les ressources nécessaires...\,»

«\,Vous aurez les ressources\,», trancha Mercier. «\,Vous aurez tout ce dont vous avez besoin.\,»

Autour de la table, les visages avaient changé. Le scepticisme avait cédé la place à autre chose — quelque chose qui ressemblait à de l'émerveillement mêlé de terreur. Ils venaient de comprendre, tous autant qu'ils étaient, que le monde qu'ils avaient connu n'existait plus.

L'humanité n'était plus seule.
