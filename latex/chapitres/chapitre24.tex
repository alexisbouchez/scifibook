% ============================================================================
%                     CHAPITRE 24 — LES ÉTOILES
% ============================================================================

\chapter{Les Étoiles}

\bigskip

Six mois avaient passé depuis le discours de Paris.

Élise vivait maintenant dans une petite maison en Normandie, à quelques kilomètres de la côte. Elle avait quitté l'ESA — non pas en disgrâce, mais en paix. Son travail était terminé. La découverte avait été faite, le message transmis, le monde changé à jamais. Ce qui viendrait ensuite appartenait à d'autres.

Léa était venue vivre avec elle.

Ce n'était pas une réconciliation spectaculaire — pas d'embrassades larmoyantes, pas de grandes déclarations d'amour. Juste deux femmes qui réapprenaient, jour après jour, à partager le même espace. À se parler au petit-déjeuner. À regarder la télévision ensemble le soir. À être présentes l'une pour l'autre, simplement présentes, sans que le travail ou les étoiles ne viennent s'interposer.

C'était un soir d'été, un de ces soirs où le ciel de Normandie se dégageait miraculeusement pour révéler la Voie lactée dans toute sa splendeur. Élise était assise sur la terrasse, un châle sur les épaules, quand Léa la rejoignit.

«\,Tu regardes encore les étoiles\,», dit sa fille en s'asseyant à côté d'elle.

«\,Oui.\,»

«\,Tu ne t'en lasses jamais\,?\,»

Élise sourit.

«\,Non. Je crois que c'est impossible de s'en lasser.\,»

Elles restèrent silencieuses un moment, côte à côte, contemplant l'immensité au-dessus de leurs têtes. Des milliers d'étoiles brillaient — chacune un soleil, peut-être entouré de planètes, peut-être abritant des mondes où personne ne vivait plus depuis des milliards d'années.

«\,Maman\,?\,»

«\,Oui\,?\,»

«\,Tu crois qu'il y en a d'autres\,? Là-haut\,? D'autres comme nous\,?\,»

C'était la question que tout le monde posait maintenant. Depuis le discours, depuis que l'humanité savait ce que contenait l'Objet de Thulé, la question revenait sans cesse : sommes-nous vraiment seuls\,? Y a-t-il encore un espoir\,?

Élise réfléchit avant de répondre.

«\,Je ne sais pas\,», dit-elle finalement. «\,Les Créateurs ont cherché pendant dix millions d'années et n'ont trouvé personne. Mais l'univers est grand. Plus grand que tout ce que nous pouvons imaginer. Peut-être que quelque part, très loin, une autre civilisation regarde le ciel en ce moment même et se pose la même question.\,»

«\,Et si on est vraiment seuls\,?\,»

Élise se tourna vers sa fille. Dans la lumière des étoiles, le visage de Léa ressemblait étrangement à celui de Marc — les mêmes yeux sombres, la même expression pensive.

«\,Si on est seuls...\,» Elle prit la main de Léa, la serra doucement. «\,Alors chaque moment comme celui-ci compte infiniment. Toi et moi, ici, à regarder les étoiles. Si personne d'autre ne les regarde en ce moment, alors c'est à nous de le faire. Et c'est... c'est quelque chose de précieux.\,»

Léa ne répondit pas tout de suite. Elle regardait le ciel, les yeux brillants, absorbant les paroles de sa mère.

«\,Tu sais quoi\,?\,» dit-elle enfin. «\,Avant, je détestais que tu regardes les étoiles. J'avais l'impression qu'elles te volaient à moi. Que tu préférais le ciel à ta propre fille.\,»

«\,Je sais. Je suis désolée.\,»

«\,Mais maintenant...\,» Léa serra la main de sa mère en retour. «\,Maintenant je comprends un peu mieux, je crois. Ce n'est pas que tu préférais les étoiles. C'est que tu avais besoin de savoir. De comprendre.\,»

«\,Oui.\,»

«\,Et maintenant que tu sais\,?\,»

Élise réfléchit. Que savait-elle, vraiment\,? Que l'univers était vaste et probablement vide. Que la conscience était rare et précieuse. Que des civilisations entières avaient existé et disparu avant même que la Terre ne se forme. Que l'humanité portait peut-être sur ses épaules la responsabilité d'être les derniers témoins du cosmos.

Mais elle savait aussi autre chose. Quelque chose de plus simple, de plus humain.

«\,Maintenant que je sais...\,» Elle regarda sa fille, ce visage qui portait les traces de Marc et les siennes, cette jeune femme qui avait grandi malgré ses absences. «\,Je sais que ce qui compte vraiment, ce n'est pas là-haut. C'est ici. C'est toi. C'est chaque instant que nous passons ensemble.\,»

Le silence retomba entre elles — un silence confortable, partagé, plein de tout ce qui n'avait pas besoin d'être dit.

Au-dessus d'elles, les étoiles brillaient. Des milliards de soleils, des trillions de mondes possibles, un univers si vaste que l'esprit humain ne pouvait en saisir qu'une infime fraction. Et quelque part, très loin, au-delà de l'orbite de Neptune, dans le froid absolu de la Ceinture de Kuiper, un objet noir continuait de tourner en silence — gardien des morts, messager des vivants, témoin éternel de tout ce qui avait été et de tout ce qui pourrait encore être.

Élise serra la main de sa fille.

«\,Léa\,?\,»

«\,Oui\,?\,»

«\,Merci d'être là.\,»

Léa posa sa tête sur l'épaule de sa mère.

«\,Merci d'être revenue.\,»

Elles restèrent ainsi, immobiles, tandis que la nuit s'épaississait et que les étoiles, indifférentes et magnifiques, poursuivaient leur danse éternelle.

Et quelque part, dans les profondeurs de l'univers, les Créateurs dormaient leur dernier sommeil, sachant que leur message avait été reçu.

L'humanité n'était plus seule avec sa solitude.

Elle savait, maintenant, ce qu'elle valait.

\bigskip

\begin{center}
\textit{FIN}
\end{center}
