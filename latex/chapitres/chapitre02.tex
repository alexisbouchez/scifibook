% ============================================================================
%                    CHAPITRE 2 — LA VÉRIFICATION
% ============================================================================

\chapter{La Vérification}

\bigskip

Les trois jours qui suivirent furent les plus longs de la vie d'Élise.

Elle avait quitté l'Observatoire à l'aube, les yeux brûlants de fatigue, et s'était traînée jusqu'à son appartement où Léa dormait encore. Elle avait dormi quatre heures d'un sommeil agité, peuplé de rêves où d'immenses sphères noires dérivaient dans le vide, puis s'était réveillée avec cette certitude lancinante qui ne la quitterait plus : elle avait vu quelque chose que personne n'avait jamais vu auparavant.

Mais voir n'était pas prouver.

La communauté scientifique, Élise le savait mieux que quiconque, était un temple exigeant. On n'y entrait pas les mains vides, armé seulement d'intuitions et de données préliminaires. Il fallait des preuves — irréfutables, reproductibles, inattaquables. Il fallait avoir épuisé toutes les alternatives avant d'oser suggérer l'impossible. Trop de carrières s'étaient brisées sur l'écueil du sensationnalisme, trop de réputations avaient sombré dans le ridicule pour qu'elle prît le risque de s'avancer sans certitudes.

Elle retourna donc à l'Observatoire, puis au centre de calcul de l'ESA, puis aux archives des missions d'exploration, traquant méthodiquement chaque explication naturelle susceptible de rendre compte de l'anomalie. Elle passa en revue les trajectoires de tous les objets connus de la Ceinture de Kuiper. Elle consulta les données des sondes Voyager, de New Horizons, de Pioneer — ces pionniers de l'exploration qui avaient, les premiers, tracé la carte des territoires lointains. Elle recalcula cent fois les équations de la mécanique céleste, variant les paramètres, testant les hypothèses, cherchant la faille qui lui aurait permis de conclure à une erreur.

Il n'y avait pas de faille.

Le quatrième jour, elle osa enfin solliciter l'avis d'un collègue. Marcus Chen, cosmologiste d'origine taïwanaise installé à Paris depuis vingt ans, était l'un des rares esprits que Élise respectait sans réserve. Il avait la rigueur d'un mathématicien et l'intuition d'un poète — combinaison rare qui lui avait valu, à cinquante-huit ans, une chaire au Collège de France et une réputation d'incorruptible.

Elle lui soumit les données sans commentaire, désignant simplement la perturbation gravitationnelle d'un geste neutre. Marcus examina les chiffres en silence pendant de longues minutes, ses sourcils broussailleux se fronçant progressivement au-dessus de ses lunettes à monture d'écaille.

«\,D'où viennent ces mesures\,?\,» demanda-t-il enfin.

«\,Horizon-7. Accéléromètres de précision. J'ai vérifié trois fois la calibration.\,»

«\,Et tu as écarté les perturbations connues\,?\,»

«\,Toutes. Ce n'est pas Éris, ce n'est pas Sedna, ce n'est rien de catalogué.\,»

Marcus se leva, fit quelques pas vers la fenêtre qui donnait sur le jardin intérieur de l'Observatoire, puis revint s'asseoir avec cette lenteur délibérée qu'Élise lui connaissait bien. C'était le signe qu'il réfléchissait intensément — qu'il pesait chaque mot avant de le prononcer.

«\,L'orbite est suspecte\,», dit-il enfin. «\,Cette circularité...\,»

«\,Je sais.\,»

«\,Les objets naturels n'ont pas d'orbites circulaires. Pas à cette distance, pas avec cette précision. La perturbation gravitationnelle des géantes gazeuses, les résonances avec Neptune... Il devrait y avoir de l'excentricité.\,»

«\,Je sais\,», répéta Élise.

Leurs regards se croisèrent. Marcus avait compris — elle le voyait dans ses yeux, cette lueur qui oscillait entre l'émerveillement et l'effroi. Il avait compris ce qu'impliquaient ces données, ce qu'elles suggéraient, ce qu'elles affirmaient presque. Mais il était trop prudent, trop conscient des pièges de l'enthousiasme, pour le dire à voix haute.

«\,Il te faut plus\,», dit-il simplement. «\,Beaucoup plus. Des observations directes. Une confirmation indépendante. Tu ne peux pas présenter ça au conseil avec seulement des perturbations gravitationnelles.\,»

«\,Je sais.\,»

«\,On va te demander si ce n'est pas une erreur instrumentale. On va te demander si tu n'as pas mal interprété les données. On va chercher toutes les raisons possibles de ne pas te croire, parce que ce que tu suggères...\,» Il s'interrompit, secoua la tête. «\,C'est trop gros, Élise. C'est le genre de découverte qui change tout.\,»

Elle hocha la tête, consciente de la justesse de ses paroles. Marcus n'avait fait que formuler ce qu'elle savait déjà : avant de pouvoir parler, il lui fallait des preuves si solides que même le plus sceptique des esprits ne pourrait les récuser. Il lui fallait transformer l'impossible en incontestable.

«\,J'ai demandé du temps d'observation sur Kepler-III\,», dit-elle. «\,Le télescope spatial. Imagerie directe dans l'infrarouge.\,»

«\,Combien de temps\,?\,»

«\,Trois semaines avant d'avoir les créneaux. Peut-être quatre.\,»

Marcus acquiesça lentement. Trois semaines. Trois semaines à garder le secret, à ronger son frein, à continuer ses vérifications dans la solitude de son bureau tandis que le reste du monde vaquerait à ses occupations ordinaires, ignorant qu'aux confins du système solaire, quelque chose d'extraordinaire attendait d'être découvert.

«\,Ne parle à personne d'autre\,», conseilla-t-il en se levant. «\,Pas encore. Pas avant d'avoir les images.\,»

«\,Je n'avais pas l'intention de—\,»

«\,Je sais.\,» Il posa une main brève sur son épaule — un geste rare, presque paternel. «\,Mais je te connais, Élise. Tu es brillante, mais tu es aussi... impatiente. Tu veux que l'univers te donne ses réponses maintenant, tout de suite. Or l'univers ne fonctionne pas ainsi. Il faut lui laisser le temps de parler.\,»

Elle le regarda partir, puis se retourna vers son écran où les données continuaient de luire, impassibles. Marcus avait raison, bien sûr. Il avait toujours raison. Mais l'attente, cette attente interminable pendant que la plus grande découverte de l'histoire humaine flottait là-haut, invisible et silencieuse, serait une torture.

Elle programma les observations, vérifia une dernière fois les coordonnées, puis rentra chez elle retrouver une fille qui ne l'attendait plus.
