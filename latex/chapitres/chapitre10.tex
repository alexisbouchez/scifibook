% ============================================================================
%                      CHAPITRE 10 — L'ACCIDENT
% ============================================================================

\chapter{L'Accident}

\bigskip

Deux jours de travail acharné avaient permis de cartographier le motif dans son intégralité.

C'était une structure d'une complexité stupéfiante : des milliers de lignes entrelacées, formant des spirales, des polygones, des formes qui semblaient osciller entre la géométrie euclidienne et quelque chose de plus étrange. Élise avait passé des heures à l'analyser, cherchant des patterns, des récurrences, un sens caché dans cet enchevêtrement de courbes.

Et elle avait trouvé quelque chose.

«\,C'est une carte\,», annonça-t-elle à l'équipage réuni dans le module de commandement. «\,Regardez ici — ces cercles concentriques, avec un point central et huit anneaux autour. C'est une représentation du système solaire.\,»

Elle désigna l'écran où le motif s'affichait, annoté de ses propres observations.

«\,Le point central, c'est le Soleil. Les huit cercles, ce sont les planètes. Et cette marque, ici, sur le huitième anneau...\,» Elle pointa une petite rosace, similaire au motif principal mais en miniature. «\,C'est nous. C'est l'Objet. Ils ont indiqué leur position.\,»

Vasquez examina l'image avec scepticisme.

«\,C'est une interprétation. On pourrait voir n'importe quoi dans ces lignes.\,»

«\,Non\,», répondit Élise. «\,Les proportions correspondent. Les distances relatives entre les cercles matchent les distances réelles entre les planètes, avec une précision de moins de deux pour cent. Ce n'est pas une coïncidence.\,»

«\,Et le reste\,?\,» demanda Okonkwo. «\,Toutes ces autres figures, qu'est-ce qu'elles représentent\,?\,»

Élise hésita.

«\,Je ne sais pas encore. Certaines ressemblent à des équations, d'autres à des schémas structurels. Il faudrait des mois pour tout déchiffrer. Mais je pense que la clé est ici —\,» elle pointa un motif particulier, une spirale complexe qui se répétait à plusieurs endroits, «\,cette figure revient constamment. Elle pourrait être un code, une séquence d'activation.\,»

«\,Activation de quoi\,?\,» demanda Vasquez.

«\,De l'entrée, peut-être. De la porte.\,»

Le mot resta suspendu dans l'air. Une porte. L'idée qu'on puisse entrer dans l'Objet, qu'il y ait un intérieur à explorer, semblait à la fois évidente et terrifiante.

«\,Il faut tester\,», dit Élise. «\,Projeter la séquence sur le motif et voir ce qui se passe.\,»

«\,Et si ce qui se passe est dangereux\,?\,» objecta Vasquez. «\,On ne sait pas ce qu'il y a là-dedans. On ne sait même pas si "là-dedans" existe.\,»

«\,C'est pour ça qu'on est venus, Commandante.\,»

\bigskip

L'EVA fut programmée pour le lendemain.

Kowalski, accompagné cette fois d'Élise elle-même, devait se rendre jusqu'au motif et tester la théorie. Ils emportaient un projecteur holographique capable de reproduire n'importe quelle séquence lumineuse — si la spirale récurrente était bien un code, ils pourraient la projeter sur la surface et observer la réaction.

Tout se passa bien pendant les premières heures. Ils atteignirent le motif sans incident, positionnèrent le projecteur, vérifièrent les alignements. Élise sentait son cœur battre à tout rompre tandis qu'elle programmait la séquence.

«\,Prêts\,?\,» demanda-t-elle.

«\,Prêts\,», confirma Kowalski.

«\,Projection dans trois, deux, un...\,»

La lumière jaillit du projecteur, traçant dans le vide la spirale complexe qu'Élise avait identifiée. Le faisceau frappa la surface noire de l'Objet, et pendant un instant, rien ne se produisit.

Puis la surface bougea.

Pas entièrement — juste une section, un cercle d'environ trois mètres de diamètre qui semblait s'enfoncer légèrement, comme si la matière se réorganisait sous leurs yeux. Un grondement sourd, transmis par le contact de leurs bottes avec la surface, fit vibrer leurs combinaisons.

«\,Ça fonctionne\,!\,» s'exclama Élise. «\,Kowalski, vous voyez ça\,?\,»

Mais Kowalski ne répondit pas. Il était en train de reculer, ses propulseurs crachant du gaz à pleine puissance, le visage déformé par la panique derrière la visière de son casque.

«\,Quelque chose m'a touché\,!\,» cria-t-il. «\,Ma jambe — quelque chose—\,»

Élise se retourna et vit. Un appendice noir, fin comme un câble, avait émergé de la surface et s'était enroulé autour de la cheville de Kowalski. Il tirait, l'attirant vers le cercle qui continuait de s'ouvrir.

«\,ARIA, urgence EVA\,!\,» hurla-t-elle. «\,Kowalski est en détresse\,!\,»

Elle s'élança vers lui, attrapa son bras, tira de toutes ses forces. Le câble noir résistait, incroyablement fort pour sa finesse. Kowalski hurlait — de peur ou de douleur, elle ne savait pas.

Puis, aussi soudainement qu'il était apparu, le câble se rétracta. Kowalski fut projeté en arrière, sa jambe libérée, et Élise le rattrapa de justesse avant qu'il ne dérive dans le vide.

«\,Je vous ramène\,», dit-elle. «\,Tenez bon.\,»

Le retour au \textit{Thulé} fut un cauchemar de quinze minutes. Kowalski perdait du sang — le câble avait percé sa combinaison, entaillé sa chair à travers plusieurs couches de matériau supposé indestructible. Okonkwo les attendait au sas, le matériel médical prêt, et prit immédiatement en charge le blessé.

«\,Ce n'est pas profond\,», annonça-t-elle après examen. «\,La combinaison a absorbé le plus gros. Mais il va falloir des points de suture et du repos.\,»

Vasquez se tourna vers Élise, le visage dur.

«\,C'est terminé\,», dit-elle. «\,Plus d'EVA. Plus d'expériences. Nous rentrons.\,»

«\,Commandante—\,»

«\,J'ai dit : nous rentrons. Cette chose est dangereuse. Elle a failli tuer un membre de mon équipage.\,»

«\,Elle ne l'a pas tué\,», répondit Élise, la voix tremblante d'adrénaline et de frustration. «\,Elle l'a relâché. Quand j'ai tiré, elle a lâché prise. Ce n'était pas une attaque — c'était un contact.\,»

«\,Un contact qui a percé une combinaison de classe A et blessé mon ingénieur.\,»

«\,Un contact qui prouve qu'il y a quelque chose là-dedans\,!\,» Élise sentait la colère monter en elle — contre Vasquez, contre l'Objet, contre l'univers entier qui lui offrait des réponses puis les lui arrachait. «\,Nous venons d'activer une ouverture. Une porte. C'est exactement ce que nous cherchions.\,»

Les deux femmes s'affrontèrent du regard. Autour d'elles, le silence du vaisseau pesait comme un reproche.

Ce fut Kowalski qui brisa l'impasse. Depuis la table médicale où Okonkwo finissait de le panser, il leva une main faible.

«\,Donnez-moi vingt-quatre heures\,», dit-il. «\,Et je retourne là-bas.\,»
